\documentclass[aspectratio=43]{beamer}
\usepackage{../../resources/agdsncourse}


\title[Linux Essentials  - Kapitel 2 - Erste Schritte mit Linux]{Linux Essentials}
\subtitle{Kapitel 2 - Erste Schritte mit Linux}
\author{Alexander Köhler}
\date{17. Oktober 2015}
\setbeamertemplate{navigation symbols}{} 
\begin{document}

% Start Folie
\logoframe

%Title
\frame{\titlepage}


%Gliederung
\setcounter{tocdepth}{1}
\section[Gliederung]{}
\frame{\tableofcontents}


%%%%%%%%%%%%%%%
\section{Open-Source-Software und Lizenzen}
\subsection{Begriffserklärungen}

\begin{frame}{Geistiges Eigentum}
  \begin{block}{Urheberrecht}
    \begin{itemize}
      \item automatisch übertragen
      \item konkrete Umsetzung einer Idee
      \item Verwertungsrecht
      \item Recht, als ``Schöpfer'' des Werkes genannt zu werden (nicht abtretbar)
    \end{itemize}
  \end{block}
  \begin{columns}
    \begin{column}{.3\textwidth}
      \begin{block}{Patent}
        \begin{itemize}
          \item Idee/Erfindung
          \item eintragbar
        \end{itemize}
      \end{block}
    \end{column}
    \begin{column}{.7\textwidth}
      \begin{block}{Markenzeichen/-recht}
        \begin{itemize}
          \item Name (Zeichen)
          \item eintragbar oder durch lange Verwendung (öffentlich anerkannt)
        \end{itemize}
      \end{block}
    \end{column}
  \end{columns}
\end{frame}

% \subsection{Lizenzen}
\begin{frame}{Warum gibt es Lizenzen?}
  \begin{itemize}
    \item Besitzer (Käufer, rechtmäßiger Herunterlader) einer Software darf sie installieren und ausführen 
    \item Urheberrecht: Werk (Programm, Softwarepaket) nicht ohne ausdrückliches Einverständnis des Urhebers vervielfältigbar
    \item Lizenz = eine Erlaubnis, Dinge zu tun, die ohne diese verboten sind
  \end{itemize}
\end{frame}

\begin{frame}{Freie Software -- Die vier Freiheiten}
  \begin{block}{Free Software Foundation (FSF)}
        ``free as in speech, not as in beer''\\
    Freie Software muss:
    \begin{enumerate}
      \item[0] für alles benutzbar
      \item[1] lesbar/verstehbar und anpassbar für eigene Zwecke 
      \item[2] Weitergabe erlauben
      \item[3] Verbesserungen erlaubt und dürfen der Allgemeinheit veröffentlicht werden
    \end{enumerate}
    $\rightarrow$ \textbf{Vier Freiheiten der Free Software Foundation}
        \flushright \includegraphics[width=0.45\textwidth]{pix/fsf-logo}
  \end{block}
\end{frame}

\begin{frame}{Offene Software -- Open Source}
  \begin{block}{Open Source Initiative (OSI)}
    \begin{itemize}
      \item Definition Open Source ähnlich freier Software
      \item für FSF zu aufgeweicht
    \end{itemize}
    \flushright \includegraphics[width=.12\textwidth]{pix/osi-logo}
  \end{block}
  \begin{block}{Allg. Zusammenfassung}
    \begin{itemize}
      \item FOSS = free and open-source software
      \item FLOSS = free, libre and open-source software
    \end{itemize}
  \end{block}
\end{frame}

\begin{frame}{}
  \begin{block}{Der Gegenpol -- proprietäre Software}
    \begin{itemize}
      \item proprietär (lat.: Eigentum) ist Gegenteil zu frei
      \item allg. für Eigenentwicklungen (Protokolle, Software, \dots)
      \item EULA (End-user license agreement)
      \item Unveränderlichkeit der Software durch Dritte
    \end{itemize}
  \end{block}
\end{frame}



\begin{frame}{Geld verdienen mit freier Software}
  \begin{columns}
    \begin{column}{0.7\textwidth}
      \begin{block}{Geschäftsmodelle}
        \begin{itemize}
          \item Zusatzleistungen (Unterstützung, Dokumentation, Training) kostenpflichtig anbieten
          \item kundenspezifische Weiterentwicklungen oder Erweiterungen auf Anfrage erstellen
          \item Softwarepaket zweimal anbieten: als FOSS-Version mit nur Grundfunktionalität und als 
              kostenpflichtige, proprietäre Version mit allen Extras
          \item \dots
        \end{itemize}
      \end{block}
    \end{column}
    \begin{column}{0.2\textwidth}
      \includegraphics[width=\textwidth]{pix/linup_front}\\
    \end{column}
  \end{columns}
\end{frame}


\subsection{General Public License (GPL)}
\begin{frame}
  \begin{block}{General Public License (GPL)}
    \begin{itemize}
      \item Copyleft-Lizenz (Software bleibt unter dieser Lizenz)
      \begin{itemize}
        \item Quellcode verfügbar und für bel. Zwecke nutzbar
        \item Veränderungen und Weitergabe dieser und des Originalcodes erlaubt, wenn der Empfänger die GPL einhält
        \item bei Weitergabe in ausführbarer Form (auch Verkauf) muss der Quellcode zugänglich sein
        \item abgeleitetes Werk: Programm, dass Teile von GPL-Programmen enthält muss ebenfalls 
              unter GPL stehen
      \end{itemize}
      $\rightarrow$ Weitergabe und Veränderung der Software
      \item Linux-Kernel zu Teilen unter GPLv2
    \end{itemize}
  \end{block}
  \begin{block}{GPLv3}
    Präzisiert Softwarepatente, Kompatibilität mit anderen freien Lizenzen
  \end{block}
\end{frame}

\subsection{Weitere Lizenzen}
\begin{frame}{Weitere Lizenzen I}
  \begin{block}{BSD-Lizenz (Berkeley Software Distribution)}
    \begin{itemize}
      \item Haftung wird ausgeschlossen
      \item Empfänger darf mit der Software machen, was er will; muss sich aber vom 
            ursprünglichen Author distanzieren
      \item keine Copyleft-Lizenz
      \item Quellcode muss nicht in der Öffentlichkeit verbleiben\\
      $\rightarrow$ kann Closed Source werden
    \end{itemize}
  \end{block}
\end{frame}

\begin{frame}{Weitere Lizenzen II}
\begin{block}{Apache-Lizenz}
  \begin{itemize}
    \item ähnlich BSD keine Copyleft-Lizenz
    \item Klauseln zur Nutzung von Patenten und Markenzeichen
    \item z.B.: Apache-Webserver, Android
  \end{itemize}
\end{block}
\begin{block}{Mozilla Public License (MPL)}
  \begin{itemize}
    \item  Mischung von GPL und BSD-Lizenz
    \item  ``schwache Copyleft-Lizenz'': MPL-Code bleibt unter MPL, 
          neuer Code auch unter anderer Lizenz\\
    $\rightarrow$ Mehrfachlizenzierung
    \item z.B.: Firefox und Thunderbird von Mozilla
  \end{itemize}
\end{block}
\end{frame}

\begin{frame}{Lizenzen für alles außer Software}
\begin{block}{}
  \begin{itemize}
    \item Creative-Commons-Lizenzen
    \begin{itemize}
      \item Bücher, Bilder, Musik, Filme, \dots
      \item Urheber legt Rahmenbedingungen fest: kommerzielle Nutzung, Änderungen
    \end{itemize}
    \item Public Domain (gemeinfrei):
    \begin{itemize}
      \item 70 Jahre nach Ableben des Urhebers
      \item in Deutschland nicht direkt zugänglich, aber im Angelsächsischen
    \end{itemize}
    \item Mehrfachlizenzierung durch Author z.B. für Programmbibliotheken
  \end{itemize}
\end{block}
\end{frame}

% \section{Prüfungsziele}
\begin{frame}{Prüfungsziele}
  \begin{alertblock}{1.3 Open-Source-Software und Lizenzen verstehen}
    \begin{description}
      \item[Gewicht]  1
      \item[Beschreibung] Open-Souce-Communities und die Lizenzierung freier und Open-Source-Software
    \end{description}
       Wissensgebiete\\ 
        \begin{itemize}
          \item Lizenzen (GPL, BSD, CC)
          \item Free Software Foundation, Open Source Initiative
          \item Free Software, Open Software, FOSS, FLOSS
          \item Open-Source-Geschäftsmodelle
        \end{itemize}
  \end{alertblock}
\end{frame}

%%%%%%%%%%%%%%%%%
\section{Erste Schritte mit Linux}
\subsection{Login and Logout}
\begin{frame}
  \begin{block}{Graphische An- und Abmeldung}
    \begin{itemize}
      \item intuitiv
      \item automatische Anmeldung sicherheitskritisch
    \end{itemize}
  \end{block}
  \begin{block}{An- und Abmelden auf der Textkonsole (Terminal)}
    \begin{itemize}
      \item z.B. lokal an virtuellen Terminal (über \taste{Strg}+ \taste{Alt}+ \taste{F1} \dots \taste{F6})
      \item Anmeldeprompt:\\
        \texttt{rechnername login:\_ }\\
        \texttt{Password:\_}\\
      \item Abmelden\\
        \texttt{\$ logout}\\
        (oder durch \taste{Strg}+ \taste{d})
    \end{itemize}
  \end{block}
\end{frame}

\subsection{Die graphische Oberfläche}
\begin{frame}{Die graphische Oberfläche}
  \begin{block}{Arbeitsumgebungen}
    \begin{itemize}
      \item große Desktopumgebungen: KDE und GNOME
      \item kleinere Alternativen: LXDE und XFCE
      \item reine Fenstermanager: Enlightenment, awesome, Blackbox, Openbox, Fluxbox, IceWM, \dots
    \end{itemize}
  \end{block}
  \begin{block}{Allgemeine Merkmale}
    \begin{itemize}
      \item Steuerleiste, Panel, Menüleiste, Startknopf
      \item Dateimanager (Dolphin, Nautilus)
      \item mehrere virtuelle Arbeitsflächen
    \end{itemize}
  \end{block}
\end{frame}

\begin{frame}{Programme der graphischen Oberfläche}
  \begin{block}{Browser}
    \begin{itemize}
      \item Firefox bzw. Iceweasel (Debian GNU Linux)
      \item Google Chrome (Open-Source-Variante von Chromium)
      \item zu finden im Menü unter ``Internet''
    \end{itemize}
  \end{block}
  \begin{block}{Terminal \& Shell}
    \begin{itemize}
      \item Standardshell: Bash (Prüfung und Vorlesung)
      \item KDE: Konsole
      \item zu finden im Menü unter ``System''
    \end{itemize}
  \end{block}
\end{frame}
\subsection{Textdateien anlegen und bearbeiten}
\begin{frame}{Vielfalt der Editoren}
  \begin{columns}
    \begin{column}{.5\textwidth}
        \begin{block}{Graphische Texteditoren}
    \begin{itemize}
      \item Kate (KDE), gedit (Gnome)
      \item selbsterklärend
    \end{itemize}
  \end{block}
    \end{column}
    \begin{column}{.5\textwidth}
      \begin{block}{Textbasierte Editoren}
        \begin{itemize}
          \item Vi bzw. Vim (LPIC 1)
          \item Emacs, Ed, JED, mcedit, elvis, \dots 
        \end{itemize}
      \end{block}
    \end{column}
  \end{columns}
  \begin{block}{nano}
    \begin{itemize}
      \item Klon des Editors pico (aus E-Mail-Paket PINE)
      \item Aufruf mit:\\
        \co{\$ nano dateiname}
    \end{itemize}
  \end{block}
\end{frame}

\begin{frame}{Tastaturkommandos von Nano}
  \begin{columns}
    \begin{column}{0.5\textwidth}
      \begin{block}{Dateien}
        \begin{tabular}{ll}
          \taste{Strg}+ \taste{o}      & Datei öffnen\\
          \taste{Strg}+ \taste{r}      & Datei einlesen\\
          \taste{Strg}+ \taste{t}      & Dateibrowser\\
        \end{tabular}
      \end{block}
    \end{column}
    \begin{column}{0.5\textwidth}
      \begin{block}{Sonstiges}
        \begin{tabular}{ll}
          \taste{Strg}+ \taste{x}      & Nano beenden\\
          \taste{Strg}+ \taste{w}      & Text suchen\\
          \taste{Strg}+ \taste{g}      & Onlinehilfe\\
        \end{tabular}
      \end{block}
    \end{column}
  \end{columns}
  \begin{block}{Dateien bearbeiten}
  \begin{tabular}{ll}
    \taste{Entf}, \taste{$\leftarrow$}  & Löschen \\
    \taste{Strg}+ \taste{k}      & Zeile ausschneiden und in Puffer verschieben\\
    \taste{Strg}+ \taste{u}      & Pufferinhalt an Cursorposition  einfügen\\
    \taste{Strg}+ \taste{\^}     & Markieren starten (alt. \taste{Strg}+ \taste{a})\\
  \end{tabular}
  \end{block}
  
\end{frame}

\begin{frame}{Aufgaben}
  \begin{itemize}
    \item Anmelden auf der Konsole
    \item Was ist der Unterschied zwischen einem falsch eingegebenen Passwort und einem falschen Nutzernamen?
    \item Graphische Anmeldung am System
    \item Wie wird das Passwort bei der Eingabe dargestellt?
    \item Vertrautmachen mit der graphischen Oberfläche
    \item Um welche handelt es sich? (ggf. mehrere Ausprobieren)
    \item Web-Browser und Terminal finden und ausprobieren
    \item Textdateibearbeitung mit nano
    \begin{itemize}
      \item nano starten
      \item einfachen Text eingeben
      \item Text als ``testtext.txt'' speichern
      \item Bearbeiten ausprobieren
    \end{itemize}

  \end{itemize}

\end{frame}


\section{Exkurs vi}
\subsection{Einf\"uhrung}
\begin{frame} 
        \begin{block}{} 
        \begin{center}
        Sure, vi is user friendly. It's just particular about who it makes friends with.
        \end{center}
        \end{block}

        \begin{exampleblock}{} 
        \begin{center}  
        Vi has two modes. The one in which it beeps and the one in which it doesn't.
        \end{center}
        \end{exampleblock}
        \begin{alertblock}{} 
        \begin{center}  
        ...and the number of the beast is vivivi...
        \end{center}
        \end{alertblock}        
\end{frame}

\subsection{Modi}
\begin{frame} 
\begin{columns}
        \begin{column}{0.5\textwidth}
          
          \begin{block}{Kommando ­Modus} 
            \begin{itemize}
            \item Standart Modus
            \item Alle Tasten sind Kommandos wie Rückgängig, Einfügen, Bewegen
            \end{itemize}
          \end{block}
          
          \begin{block}{Kommandozeilen­ Modus} 
            \begin{itemize}
            \item Mit \taste{:} aktivieren
            \item Aufgaben wie Suchen, Ersetzen, Speicher, Laden, ...
            \item Verlassen mit \taste{esc}
            \end{itemize}
          \end{block}
          
         \end{column} 
        \begin{column}{0.5\textwidth}

          \begin{block}{Einfüge­ Modus} 
            \begin{itemize}
            \item Mit \taste{i} aktivieren (u.a.)
            \item \textit{Normales} schreiben von Text
            \item Verlassen mit \taste{esc}
            \end{itemize}
          \end{block}
          
          \begin{block}{Visueller Modus} 
            \begin{itemize}
            \item Mit \taste{v} aktivieren
            \item Text markieren an dem Operationen vorgenommen werden sollen
            \item Verlassen mit \taste{esc}
            \end{itemize}
          \end{block}
        
        \end{column} 
\end{columns}
\end{frame}

\subsection{Navigation}
\begin{frame} 
\begin{columns}
        \begin{column}{.5\textwidth}
        \begin{exampleblock}{Navigation} 
        \taste{h} links \taste{j} unten\\
        \taste{k} oben (klettern) \taste{l} rechts\\
        \taste{+} Zum ersten Zeichen der folgenden Zeile\\
        \taste{–} der vorangegangenen Zeile\\
        \taste{e}\taste{E} An das Wortende (ohne/mit .!?)\\
        \taste{w}\taste{W} Wortweise vorwärts (ohne/mit .!?)\\
        \taste{b}\taste{B} Wortweise rückwärts (ohne/mit .!?)\\

        \end{exampleblock}
        \end{column} 
        
        \begin{column}{.5\textwidth} 
        \begin{exampleblock}{} 
        \taste{0} An den Zeilenanfang\\
        \taste{\$} An das Zeilenende\\
        \taste{G} Zur letzten Zeile\\
        \taste{H} Zur ersten Zeile (Fenster) \\
        \taste{L} Zur letzten Zeile (Fenster)\\
        \taste{Strg}\taste{b}  Bildschirmseite auf\\
        \taste{Strg­}\taste{f}  Bildschirmseite ab
                \end{exampleblock}
        \end{column} 
\end{columns}
\end{frame}

\subsection{Editieren}
\begin{frame} 
\begin{exampleblock}{Editierbefehle} 
\begin{table}[htdp]
\begin{center}
\begin{tabular}{c|l|l|l}
  Ziel & Ändern & Ausschneiden & Kopieren \\\hline
  1 Wort & \textcolor{blue}{\taste{c}}\taste{w}&\textcolor{red}{\taste{d}}\taste{w}&\textcolor{green!50!black}{\taste{y}}\taste{w}\\
  2 Wörter (ohne .!?) & \textcolor{blue}{\taste{c}}\taste{2}\taste{w}&\textcolor{red}{\taste{d}}\taste{2}\taste{w}&\textcolor{green!50!black}{\taste{y}}\taste{2}\taste{w}\\
  3 Wörter rückwärts & \textcolor{blue}{\taste{c}}\taste{3}\taste{b}&\textcolor{red}{\taste{d}}\taste{3}\taste{b}&\textcolor{green!50!black}{\taste{y}}\taste{3}\taste{b}\\
  1 Zeile & \textcolor{blue}{\taste{c}}\taste{c}&\textcolor{red}{\taste{d}}\taste{d}&\textcolor{green!50!black}{\taste{y}}\taste{y}\\
  Bis Zeilenanfang & \textcolor{blue}{\taste{c}}\taste{0}&\textcolor{red}{\taste{d}}\taste{0}&\textcolor{green!50!black}{\taste{y}}\taste{0}\\
  Bis Zeilenende & \textcolor{blue}{\taste{c}}\taste{\$}&\textcolor{red}{\taste{d}}\taste{\$}&\textcolor{green!50!black}{\taste{y}}\taste{\$}\\
  1 Zeichen & \taste{r}&\taste{x}&\textcolor{green!50!black}{\taste{y}}\taste{l}\\
  5 Zeichen & \taste{5}\taste{s}&\taste{5}\taste{x}&\textcolor{green!50!black}{\taste{y}}\taste{5}\taste{l}\\
\end{tabular}
\end{center}
\label{default}
\end{table}%
\vspace{-0.5cm}
\begin{itemize}
\item \taste{p} Einfügen unter dem Cursor (dahinter)
\item \taste{P} Einfügen über dem Cursor (davor)
\end{itemize}

\end{exampleblock}

\end{frame}
\begin{frame} 
\begin{exampleblock}{Editierbefehle II} 
\begin{table}[htdp]
\begin{center}
\vspace{-0.2cm}
\begin{tabular}{p{0.8\textwidth}|c}
  Aufgabe& Taste\\\hline
  Text an der aktuellen Position einfügen&\taste{i}\\
  Text am Zeilenanfang einfügen&\taste{I}\\
  Text an der aktuellen Position anfügen&\taste{a}\\
  Text am Zeilenende anfügen&\taste{A}\\
  Unterhalb des Cursors eine neue Zeile einfügen&\taste{o}\\
  Oberhalb des Cursors eine neue Zeile einfügen&\taste{O}\\
  Zeile löschen und Text ersetzen&\taste{S}\\
  Vorhandene Zeichen mit neuem Text überschreiben&\taste{R}\\
  Aktuelle und folgende Zeile zusammenführen&\taste{J}\\
  %Groß­/Kleinschreibung umsetzen&\taste{$\sim$}\\
  Letzte Aktion wiederholen&\taste{.}\\
  Letzte Änderung zurücknehmen&\taste{u}\\
  letzte bearbeite Zeile in ihren ursprünglichen Zustand zurückversetzen&\taste{U}\\

\end{tabular}
\end{center}
\label{default}
\end{table}%
\end{exampleblock} 
\end{frame}

\begin{frame} 
\begin{columns}
        \begin{column}{0.32\textwidth}
          \begin{exampleblock}{Suchen} 
            \textcolor{gray}{\taste{:}}\taste{/}  Textabwärts suchen\\
            \textcolor{gray}{\taste{:}}\taste{?}  Textaufwärts suchen\\
            \taste{n} Textabwärts fortsetzten\\
            \taste{N} Textaufwärts fortsetzen
          \end{exampleblock}
        \end{column}
        \begin{column}{0.75\textwidth} 
          \begin{exampleblock}{Ersetzen} 
            \co{:[Start,Ende]s/reg.~Ausdr/Ersetzung[/g]}\\
            \co{:5,\$s/schwarz/blau} Nur das erste Vorkommen\\
            \co{:5,\$s/schwarz/blau/g/i} Alle Vorkommen pro Zeile\\
            \co{:\%s/Zebra/\&fink} An Funde anhängen
          \end{exampleblock}
        \end{column} 
\end{columns}
\begin{columns}
        \begin{column}{0.7\textwidth}
        \begin{exampleblock}{Markieren} 
        \taste{m}\taste{$x$} setzt Marke $x$\\
        \taste{'}\taste{$x$}  zurück zu Marke $x$\\
        \taste{'}\taste{'} zurück zur letzten Stelle vom Sprung\\
        \taste{'} zur letzten Zeile (Anfang) vom Sprung\\
        \taste{:}\co{marks} alle Marken auflisten
        \end{exampleblock}
        \end{column} 
        \begin{column}{0.3\textwidth} 
        \begin{exampleblock}{Marken} 
        \co{a-z} Lokale Marken\\
        \co{A-Z} Globale Marken\\
        \co{0-9} Bleibende globale Marken\\
        \end{exampleblock}              
        \end{column} 
\end{columns}
\end{frame}

\begin{frame} 
\begin{exampleblock}{ex Kommandos} 
% \hspace{-2em}
% \begin{table}[htdp]
% \begin{center}
\begin{tabular}{l|p{0.6\textwidth}}
\taste{:}\taste{w}\co{Datei}&schreibt den kompletten Pufferinhalt in \co{Datei}\\
\taste{:}\taste{w}\taste{!}\co{Datei}&schreibt auch in schreibgeschützte \co{Datei}\\
\taste{:}\taste{e}\co{Datei}& liest die angegebene \co{Datei} in den Puffer\\
\taste{:}\taste{e}\taste{\#}& liest die letzte \co{Datei} in den Puffer\\
\taste{:}\taste{r}\co{Datei}& fügt den Inhalt von \co{Datei} hinter aktueller Zeile ein\\
\taste{:}\taste{!}\co{Kommando}& führt das \co{Kommando} aus und kehrt zu vi zurück\\
\taste{:}\taste{r}\taste{!}\co{Kommando}& fügt Ausgabe von \co{Kommando} hinter aktueller Zeile ein\\
\taste{:}\taste{q}& beendet vi \\
\taste{:}\taste{q}\taste{!}& beendet vi ohne Rückfragen\\
\taste{:}\taste{x} oder & sichert und beendet vi\\
\taste{:}\taste{w}\taste{q} \\
\end{tabular}
% \end{center}
% \end{table}%
\end{exampleblock}      
\end{frame}



%%%%%%%%%%%%%%%%%%%%%%%%%% next time...
\begin{frame}[plain]
  \begin{alertblock}{Nächste Vorlesung}
    \textbf{Termin:} Nach der Mittagspause\\
    \textbf{Thema:} Kapitel 3 - Die Shell \\
    \textbf{Lehrbuchkapitel:} 
    \begin{itemize}
      \item LXES 4 Computer, Software und Betriebssysteme
      \item LXES 5 Linux und freie Software
    \end{itemize}
  \end{alertblock}
\end{frame}



\materialframe

\end{document}
