\documentclass[aspectratio=43]{beamer}
\usepackage{../../resources/agdsncourse} 

\title[Linux Essentials  - Kapitel 9 - Zugriffsrechte]{Linux Essentials}
\subtitle{Kapitel 9 - Zugriffsrechte}
\author{Hagen Eckert}
\date{25. Oktober 2015}


\begin{document}

% Start Folie
\logoframe

%Title
\frame{\titlepage}

%Gliederung
\setcounter{tocdepth}{1}
\section[Gliederung]{}
\frame{\tableofcontents}
% Inhalt
%%%%%%%%%%%%%%%
%%%%%%%%%%%%%%%%%%%%%%%%%%%%
\section{Zugriffsrechte}
\subsection{Konzept}
\begin{frame} 
	\begin{block}{Konzept} 
	\begin{itemize}
	\item Mehrbenutzersystem
	\item Abgrenzung zwischen normalen Benutzern und Administratoren
	\item Abschottung der Daten einzelner Nutzer untereinander
	\item Unterstützung von Gruppen und allgemeinen Rechten
	\end{itemize}
	\end{block}


\end{frame}

\subsection{Rechtematrix}

\begin{frame} 
	\begin{block}{Rechtematrix} 
 \begin{align}
\bordermatrix{
 &r	& w   & x      \cr
u & 1 & 1 & 1  \cr
g & 1 & 0 & 1  \cr
o & 1 & 0 & 0  \cr
}\qquad
\bordermatrix{
 &r	& w   & x      \cr
u & 4 & 2 & 1  \cr
g & 4 & 0 & 1  \cr
o & 4 & 0 & 0  \cr
}\nonumber
 \end{align}
	\begin{itemize}
	\item Nutzer (U), Gruppe (G), Andere (O)
	\item Lese- (r), Schreib- (w), Ausführungsrecht (x)
	\end{itemize}
	\end{block}


\end{frame}


\begin{frame} 
	\begin{block}{Rechtematrix anzeigen und ändern}
	\begin{itemize}
	\item \co{ls}
	\item \co{chmod}
	\end{itemize}
	\end{block}
	\begin{exampleblock}{}
	\begin{itemize}
	\item \co{ls -l}
	\item \co{chmod u+x <file>}
	\item \co{chmod ug=rw,o=r <file>}
	\item \co{chmod 644 <file>}
	\end{itemize}
	\end{exampleblock}
\end{frame}

\subsection{Eigentümer}
\begin{frame} 
	\begin{block}{Eigentümer}
	\begin{itemize}
	\item \co{ls}
	\item \co{chown}
	\item \co{chgrp}
	\end{itemize}
	\end{block}
	\begin{exampleblock}{}
	\begin{itemize}
	\item \co{ls -l}
	\item \co{chown <user> <file>}
	\item \co{chown <user>:<group> <file>}
	\item \co{chown :<group> <file>}
	\item \co{chgrp <group> <file>}
	\end{itemize}
	\end{exampleblock}
\end{frame}
\subsection{Prozesse}

\begin{frame} 
	\begin{block}{Prozesse}
	\begin{itemize}
	\item Auch Prozesse haben eine Besitzer
	\item In der Regel durch den Ausführenden bestimmt
	\item Prozesse von root (SUID) können ihren Besitzer ändern
	\end{itemize}
	\end{block}
	\begin{exampleblock}{}
	\begin{itemize}
	\item \co{ps -ua}
	\end{itemize}
	\end{exampleblock}
\end{frame}

\begin{frame} 

\subsection{Besonderheiten}
	\begin{block}{Besonderheiten}
	\begin{itemize}
	\item Prozesse können auch dem Eigentümer ihrer Datei gehören
	\item Dies wird mit dem Set-UID und Set-GID Bit umgesetzt
	\item In der Octaldarstellung 4 (SUID) oder 2 (SGID) an 0. Stelle
	\end{itemize}
	\end{block}
	\begin{exampleblock}{}
	\begin{itemize}
	\item \co{chmod 2664 <file>}
	\item \co{chmod u+s <file>}
	\end{itemize}
	\end{exampleblock}
\end{frame}


\begin{frame} 

\subsection{Besonderheiten}
	\begin{block}{Besonderheiten}
	\begin{itemize}
	\item Bei einem Verzeichnis erzwingt ein gesetztes SGID das alle neu erstellten Dateien Unterverzeichnisse  der Gruppe des Verzeichnisses gehören
	\item Eine weiter Besonderheit ist das StickyBit
	\item In der Octaldarstellung hat es eine 1 (0. Stelle) und in als Abkürzung  $\pm$t
	\end{itemize}
	\end{block}

\end{frame}
\section{Netzwerk}
\begin{frame} 
\subsection{Protokolle}
	\begin{block}{Protokolle}
	\begin{itemize}
	\item 3 Hauptebenen:
	\begin{itemize}
	\item Übertragungsprotokolle (Ethernet, IEEE 802.11,...)
	\item Kommunikationsprotokolle (TCP/IP, UDP, ICMP, ... )
	\item Anwendunksprotokolle (http, pop3, dns, sip, ...)
	\end{itemize}
	\item Detaillierter Schichtsystem ISO/OSI (7 Ebenen)
	\end{itemize}
	\end{block}
	
\end{frame}
\subsection{Adresse und Routing}
\begin{frame} 

	\begin{block}{Adresse}
	\begin{itemize}
	\item In der Regel IPv4
	\item 4 Oktette 255.255.255.255
	\item Die IP muss nicht konstant bleiben und ein Gerät hat in der Regel mehrere gleichzeitig 
	\item Spezielle Bereiche wie 10.0.0.0/8
	\item Vergabe früher direkt in Blöcken (/8 16777241 Adressen)
	\end{itemize}
	\end{block}
	
\end{frame}

\begin{frame} 
	\begin{exampleblock}{\co{whois 3.0.0.0}}
	NetRange:       3.0.0.0 - 3.255.255.255\\
	CIDR:           3.0.0.0/8\\
	NetName:        GE-INTERNET\\
	NetHandle:      NET-3-0-0-0-1\\
	Parent:          ()\\
	NetType:        Direct Assignment\\
	Organization:   General Electric Company (GENERA-9)\\
	RegDate:        1988-02-23\\
	Updated:        2008-03-28\\
	Ref:            http://whois.arin.net/rest/net/NET-3-0-0-0-1\\

	\end{exampleblock}
\end{frame}

\begin{frame} 
	\begin{exampleblock}{\co{whois 9.0.0.0}}
	NetRange:       9.0.0.0 - 9.255.255.255\\
	CIDR:           9.0.0.0/8\\
	NetName:        IBM\\
	NetHandle:      NET-9-0-0-0-1\\
	Parent:          ()\\
	NetType:        Direct Assignment\\
	Organization:   IBM (IBM-1)\\
	RegDate:        1988-12-16\\
	Updated:        2011-09-06\\
	Ref:            http://whois.arin.net/rest/net/NET-9-0-0-0-1
	\end{exampleblock}
\end{frame}

\begin{frame} 
	\begin{exampleblock}{\co{whois 11.0.0.0 / 21.0.0.0 / 22.0.0.0 / 30.0.0.0 / ...}}
	NetRange:       30.0.0.0 - 30.255.255.255\\
	CIDR:           30.0.0.0/8\\
	NetName:        DNIC-NET-030\\
	NetHandle:      NET-30-0-0-0-1\\
	Parent:          ()\\
	NetType:        Direct Allocation\\
	Organization:   DoD Network Information Center (DNIC)\\
	RegDate:        1991-07-01\\
	Updated:        2009-06-19\\
	Ref:            http://whois.arin.net/rest/net/NET-30-0-0-0-1\\
	\end{exampleblock}
\end{frame}

\begin{frame} 
	\begin{exampleblock}{\co{whois}}
	\begin{itemize}
	\item 12.0.0.0/8 $\rightarrow$ AT\&T
	\item 13.0.0.0/8 $\rightarrow$ Xerox
	\item 15.0.0.0/8 $\rightarrow$ Hewlett-Packard
	\item 16.0.0.0/8 $\rightarrow$ Hewlett-Packard
	\item 17.0.0.0/8 $\rightarrow$ Apple
	\item 18.0.0.0/8 $\rightarrow$ Massachusetts Institute of Technology
	\item 19.0.0.0/8 $\rightarrow$ Ford Motor Company
	\item 20.0.0.0/8 $\rightarrow$ Computer Sciences Corporation
	\end{itemize}
	\end{exampleblock}
\end{frame}


\begin{frame} 

	\begin{block}{Routing}
	\begin{itemize}
	\item Freie Wegfindung
	\item Wenn ein Paket nicht direkt zugestellt werden kann wird es weitergereicht (Default-Route)
	\item Jedes Paket kann einen anderen Weg nehmen
	\item Ermöglicht Loadbalancing
	\end{itemize}
	\end{block}
	
\end{frame}

\subsection{DNS}
\begin{frame} 

	\begin{block}{DNS}
	\begin{itemize}
	\item Umwandlung leicht zu merkender Namen in IP-Adressen
	\item Zusätzliche Einträge wie Mailserver einer Domain möglich
	\item Auch eine Reverse-Anfrage ist möglich
	\item Schichtweiser Aufbau
	\end{itemize}
	\end{block}
	
\end{frame}

\subsection{IPv6}
\begin{frame} 
	\begin{block}{\co{IPv6}} 
	\begin{itemize}
	\item \textcolor{red}{2001:0db8}:\textcolor{green!50!black}{85a3}:\textcolor{blue}{08}\textcolor{orange}{d3}:\textcolor{purple}{1319:8a2e:0370:7347}
	\item Präfix: \textcolor{red}{2001:0db8:85a3:08d3}::
	\item Interface Identifier \textcolor{purple}{1319:8a2e:0370:7347}
	\item Provider Präfix \textcolor{green!50!black}{2001:0db8::/32}
	\item Endkunde Präfix    \textcolor{blue}{2001:0db8:85a3::/48} oder nur \textcolor{orange}{2001:0db8:85a3:0800::/56} 
	\end{itemize}
	\end{block}
\end{frame}


\begin{frame}[plain]
\begin{alertblock}{Nächste Vorlesung}
\textbf{Termin:} In 15 Minuten\\
\textbf{Thema:} Kapitel 10 - Mehr Netzwerk \\
\textbf{Lehrbuchkapitel:} 
\begin{itemize}
\item LXES 15 Linux im Netz 
\end{itemize}
\end{alertblock}
\end{frame}

\materialframe

\end{document}
