\documentclass[aspectratio=43]{beamer}
\usepackage{../../resources/agdsncourse} 

\title[Linux Essentials  - Kapitel 3 - Die Shell]{Linux Essentials}
\subtitle{Kapitel 3 - Die Shell}
\author{Hagen Eckert}
\date{17. Oktober 2015}


\begin{document}

% Start Folie
\logoframe

%Title
\frame{\titlepage}

%Gliederung
\setcounter{tocdepth}{1}
\section[Gliederung]{}
\frame{\tableofcontents}


% Inhalt
%%%%%%%%%%%%%%%
\section{Shell Grundlagen}
\subsection{Was ist die Shell?}
\begin{frame} 
	\begin{block}{shell} 
	\begin{itemize}
	\item Eine Schnittstelle zum Computer/Linux
	\item Durch Texteingaben werden Aktionen ausgelöst und deren Ergebnisse angezeigt
	\item An sich ein Programm wie viele andere
	\end{itemize}
	\end{block}

	


\end{frame}

\begin{frame} 

	\begin{exampleblock}{Shell Beispiele} 
	\begin{itemize}
	\item \co{sh} für die klassische Bourne-\textbf{Sh}ell (\co{sh} eigentlich \co{dash})
	\item \co{bash} für die \textbf{B}ourne-\textbf{A}gain-\textbf{Sh}ell 
	\item \co{ksh} für die \textbf{K}orn-\textbf{Sh}ell
	\item \co{csh} für die \textbf{C}-\textbf{Sh}ell (\co{csh} oft \co{tcsh})
	\item \co{tcsh} für die \textbf{T}enex-\textbf{C}-\textbf{Sh}ell, eine weiterentwickelte C-Shell
	\end{itemize}
	\end{exampleblock}
	\begin{exampleblock}{}
	\begin{itemize}
	\item \co{exit} verlässt die aktuelle Shell
	\item \co{echo \$0} liefert die aktuelle Shell zurück
	\item \co{echo \$0 --version} liefert die Version der Shell zurück
	\end{itemize}
	\end{exampleblock}


\end{frame}


\subsection{Kommandos}
\begin{frame} 

  \begin{block}{Kommandos} 
  \begin{itemize}
  \item Shell interpretiert die Eingabe nach Wörter getrennt
  \item Erstes Wort in der Regel \textit{Kommando}
  \item Wörter mit führendem \co{-} \textit{Optionen}
  \item Wörter ohne führendem \co{-} \textit{Argumente}
  \end{itemize}
  \end{block}

  \begin{block}{Optionen}
  \begin{itemize}
  \item \co{-v -l -t} wirken in der Regel wie Schalter
  \item Häufig existieren Langformen: \co{--verbose --long --tree} 
  \end{itemize}
  \end{block}

\end{frame}



\begin{frame} 

  \begin{exampleblock}{Beispiele} 
  \begin{itemize}
    \item  \textit{Kommando}: Was wird gemacht
    \item \textit{Optionen}: Wie wird es gemacht 
    \item \textit{Argumente}: Womit wird es gemacht
  \end{itemize}
  \end{exampleblock}

\end{frame}

\begin{frame}

	\begin{block}{Kommando Arten} 
	\begin{itemize}
	\item Interne Kommandos
	\begin{itemize}
		\item Um die 70 Befehle
		\item Beeinflussen oft die Shell selbst
		\item Sehr schnell
		\item z.B. \co{echo}, \co{kill}, \co{time}, \co{cd}, ... 
	\end{itemize}
	
	\item Externe Kommandos
	\begin{itemize}
		\item Werden durch ausführbare Dateien zur Verfügung gestellt
		\item z.B. in \co{/bin}, \co{/sbin}, ...
		\item Festgelegt in Umgebungsvariable \co{\$PATH}
	\end{itemize}
	
	\end{itemize}
	\end{block}

\end{frame}

\begin{frame}
	
	\begin{exampleblock}{Beispiele} 
	\begin{itemize}
	\item  \co{type echo}
	\item  \co{type date}
	\item  \co{type top}
	\item  \co{type kill}
	\item  \co{type help}
	\item  \co{type info}
	\item  \co{type man}
	\end{itemize}
	\end{exampleblock}

\end{frame}

\subsection{Shell-Variablen \& Umgebung}
\begin{frame} 

	\begin{block}{Variablen} 
	\begin{itemize}
	\item Bash $\approx$ Programmiersprache 
	\item Texte und numerische Werte können gespeichert werden
	\item Variablen dienen auch zur Steuerung der Shell
	\end{itemize}
	\end{block}

\begin{block}{Umgebung}
\begin{itemize}
\item Shellvariablen sind nur innerhalb der Shell vorhanden 
\item um auch für Programme verfügbar zu sein muss exportiert werden (\co{export bla})
\item \co{set}  zeigt alle Variablen
\item  \co{env} oder \co{export} nur exportierte
\end{itemize}

\end{block}
\end{frame}

\begin{frame} 

	\begin{exampleblock}{Beispiele} 
	\begin{itemize}
	\item \co{echo bla}
	\item \co{bla=foo}
	\item \co{echo \$bla}
	\item \co{echo \$PS1}
	\item \co{unset bla}
	\item \co{echo \$RANDOM \$SECONDS \$SHLVL \$OLDPWD \$MACHTYPE}
	\item \co{declare}
	\item \co{shopt}
	\end{itemize}
	\end{exampleblock}

\end{frame}

%%%%%%%%%%%%%%%%%
\section{Shell alias Werkzeugkoffer}
\subsection{Ein Kommando - date}
\begin{frame} 


	\begin{block}{\co{date}} 
	\begin{itemize}
	\item Wichtiges Tool wenn es um Zeit geht
	\item Kann die Softwareuhr setzen (root)
	\item Unterschiedlichste Layouts
	\item Zeitzonen 
	\end{itemize}
	\end{block}

\end{frame}

\begin{frame} 


	\begin{exampleblock}{Beispiele} 
	\begin{itemize}
	\item \co{date}
	\item \co{export TZ=Europe/London}
	\item \co{tzselect}
	\item \co{date +\'\%d. \%B \%Y\'}
	\item \co{date +\'Tag Nr. \%j\'}
	\item \co{date +\'KW Nr. \%V\'}
	\end{itemize}
	\end{exampleblock}

\end{frame}

\subsection{Autovervollständigung}
\begin{frame} 

	\begin{block}{Autovervollständigung} 
	\begin{itemize}
	\item \taste{$\leftrightharpoons\quad$}  vervollständigt Eingaben
	\item sowohl Befehle im \co{PATH}
	\item als auch Speicherorte 
	\item 2x \taste{$\leftrightharpoons\quad$} macht Vorschläge
	\item alternativ \taste{esc}\taste{?} Kommandos und \taste{esc}\taste{/} Orte
	\end{itemize}
	\end{block}

\end{frame}

\subsection{History}
\begin{frame} 

	\begin{block}{History} 
	\begin{itemize}
	\item \taste{$\;\uparrow\;$} \taste{$\;\downarrow\;$}  blättert durch die History
	\item \taste{ctrl}\taste{r} inkrementell durchsuchen
	\item \co{history} gibt alle alten Befehle aus
	\item \co{!} erlaubt Zugriff auf einzelne Befehle 
	\end{itemize}
	\end{block}

\end{frame}

\begin{frame} 

	\begin{exampleblock}{Beispiele} 
	\begin{itemize}
	\item \co{!!}
	\item \co{!-2}
	\item \co{!echo}
	\item \co{!?echo}
	\item \co{!-2:0-}
	\item \co{!-3:1-2}
	\item \co{!-3:1*}
	\end{itemize}
	\end{exampleblock}

\end{frame}

\subsection{Befehlsketten}
\begin{frame} 

	\begin{block}{Befehlsketten} 
	\begin{itemize}
	\item Befehle können mit \co{;} getrennt werden
	\item \co{\&\&} wie \co{;} wenn exitcode == 0
	\item \co{||} wie \co{;} wenn exitcode != 0
	\end{itemize}
	\end{block}

\end{frame}


\begin{frame} 

	\begin{exampleblock}{Beispiele} 
	\begin{itemize}
	\item \co{echo \'Heute ist der \' ; date +\%d.\%m.\%Y}
       \item \co{echo \'Heute ist der \' \&\& date +\%d.\%m.\%Y}
       \item \co{echo \'Heute ist der \' || date +\%d.\%m.\%Y}
	\end{itemize}
	\end{exampleblock}

\end{frame}

\section{Hilfe}
\subsection{Übersicht}
\begin{frame} 

	\begin{block}{\co{help}} 
	\begin{itemize}
	\item \co{help} ist die interne Hilfe der \co{bash}
	\item Gibt Informationen über die eingebauten Funktionen
	\end{itemize}
	\end{block}

	\begin{block}{\co{info}} 
	\begin{itemize}
	\item Sehr ausführlich 
	\item Aufbau wie eine Webseite
	\end{itemize}
	\end{block}
	
	\begin{block}{\co{man}} 
	\begin{itemize}
	\item Klassische Hilfe Seiten
	\item Aufgeteilt in Kategorien  
	\end{itemize}
	\end{block}

\end{frame}

\subsection{help}
\begin{frame} 

	\begin{exampleblock}{Beispiele \co{help}} 
	\begin{itemize}
	\item \co{help}
	\item \co{help help}
	\item \co{help echo}
	\item \co{help kill}
	\item \co{help for}
	\item \co{help date}
	\end{itemize}
	\end{exampleblock}

\end{frame}
\subsection{info}
\begin{frame} 



	\begin{exampleblock}{Beispiele \co{info}} 
	\begin{itemize}
	\item \co{info date}
	\item \co{info}
	\item \co{info dhcp}
	\end{itemize}
	\end{exampleblock}
	
	\begin{block}{\co{info} Tasten} 
	\begin{itemize}
	\item \taste{q} Beenden
	\item \taste{h} Hilfe
	\item \taste{n} Nächster Knoten
	\item \taste{p} Vorheriger Knoten
	\item \taste{u} Übergeordneter Knoten
	\end{itemize}
	\end{block}

\end{frame}

\begin{frame} 
	\begin{exampleblock}{Beispiele \co{info}} 
	\begin{itemize}
	\item \co{man date}
	\item \co{man info}
	\item \co{man dhcp}
	\item \co{man -k dhcp}
	\end{itemize}
	\end{exampleblock}
	

\end{frame}

%%%%%%%%%%%%%%%%%%%%%%%%%%
%%%%%%%%%%%%%%%%%%%%%%%%%%
\begin{frame}[plain]
\begin{alertblock}{Nächste Vorlesung}
\textbf{Termin:} Morgen, 18.10.2015 9:20 Uhr\\
\textbf{Thema:} Kapitel 4 - Dateien \\
\textbf{Lehrbuchkapitel:} 
\begin{itemize}
\item LXES 6 Datein: Aufzucht und Pflege
\end{itemize}
\end{alertblock}
\end{frame}

\materialframe
\versionframe

\end{document}
