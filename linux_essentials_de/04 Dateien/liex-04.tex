\documentclass[aspectratio=43]{beamer}
\usepackage{../../resources/agdsncourse} 

\title[Linux Essentials  - Kapitel 4 - Dateien]{Linux Essentials}
\subtitle{Kapitel 4 - Dateien}
\author{Alexander Köhler}
\date{18. Oktober 2015}
\begin{document}

%%%%%%%%%%%%%%%%%%%%%%%
% Start Folie
\logoframe

%Title
\frame{\titlepage}

%Gliederung
\setcounter{tocdepth}{1}
\section[Gliederung]{}
\frame{\tableofcontents}


%%%%%%%%%%%%%%%
\section{Dateien und Verzeichnisse}
\subsection{Dateinamen}
\begin{frame}{Dateinamen}
%   \begin{block}{Dateinamen}
    \begin{itemize}
      \item Dateien sind zusammengehörende Daten, die auf einem Speichermedium unter einem
        Namen (Dateinamen) gespeichert sind
      \item jede Datei besitzt eine eindeutige Nummer für ihr Dateisystem (Inode-Nr.)
      \item Zeichen in Dateinamen:
      \begin{itemize}
        \item Alles erlaubt, außer ``/'' und Nullbyte (ASCII-Wert 0)\\
          $\rightarrow$``/'' ist Trenner in Pfadnamen
        \item Unterscheidung zwischen Groß-/Kleinschreibung
        \item Sonderzeichen (\$, \textasteriskcentered, Umlaute, Leerzeichen) durch ``\textbackslash''oder 
        Anführungszeichen vor der Shell verstecken
        \item Umlaute hängen vom System ab (s. \co{convmv})
        \item Typische Obergrenze: 255 Zeichen
      \item Endungen (z.B. \co{.txt}) sind fakultativ
      \end{itemize}
      \item Versteckte Dateien beginnen mit ``.''
    \end{itemize}
%   \end{block}
\end{frame}

\subsection{Verzeichnisse}
\begin{frame}{Verzeichnisse}
  \begin{block}{Grundlagen zu Verzeichnissen unter Linux}
  \begin{itemize}
    \item Gruppieren Dateien hierarchisch zum besseren Überblick
    \item Gleiche Regeln für Verzeichnisnamen wie bei Dateinamen
    \item Können andere Verzeichnisse enthalten \\
      $\rightarrow$ Verzeichnisbaum
    \item Wurzel des Verzeichnisbaums ist das Wurzelverzeichnis (root~directory)
          und wird mit ``/'' bezeichnet
  \end{itemize}
  \end{block}
  \begin{block}{Konvention der Verzeichnishierarchie}
    $\rightarrow$ Siehe Kapitel 10 Linux Essentials
  \end{block}

\end{frame}

\subsection{Pfadangaben}
\begin{frame}{Pfadangaben: Der Weg zur Datei durch das Dateisystem}
  \begin{columns}
    \begin{column}{0.5\textwidth}
      \begin{block}{Absolute Pfadangaben}
        \begin{itemize}
          \item Beginnen mit ``/'' und 
          \item Beschreiben Weg zur Datei ausgehend vom Wurzelverzeichnis
        \end{itemize}
      \end{block}
    \end{column}
    \begin{column}{0.5\textwidth}
      \begin{block}{Relative Pfadangaben}
        \begin{itemize}
          \item Beginnen nicht mit ``/''
          \item Beschreiben einen Weg, der im aktuellen Verzeichnis beginnt
        \end{itemize}    
      \end{block}
    \end{column}
  \end{columns}
  \begin{block}{Wichtige Namen für Spezialverzeichnisse}
  \begin{description}
    \item[``..''] Übergeordnetes Verzeichnis
    \item[``.''] Aktuelles Verzeichnis (Nützlich für \co{./prog})
    \item[``\textasciitilde''] Heimatverzeichnis des aktuellen Nutzers
  \end{description}
  \end{block}
\end{frame}


\section{Arbeiten mit und in Verzeichnissen}
\subsection{Bewegen in Verzeichnissen}
\begin{frame}{Bewegen in Verzeichnissen}
  \begin{columns}
    \begin{column}{0.5\textwidth}
      \begin{block}{Das aktuelle Verzeichnis}
        \begin{itemize}
          \item Wichtig für relative Pfade
          \item Wird an Prozesse vererbt
          \item Ermittelbar über: \\ \co{\$ pwd} \\(print working dir)
        \end{itemize}
      \end{block}
    \end{column}
    \begin{column}{0.5\textwidth}
       \begin{block}{Arbeitsverzeichnis wechseln}
            \co{cd} (change directory):
            \begin{itemize}
              \item \co{\$ cd verzeichnis}
              \item \co{\$ cd}
              \item \co{\$ cd -}
              \item Ist interner Shell-Befehl
              \item Vgl. \co{dirs}, \co{popd},\co{pushd}
            \end{itemize}
        \end{block}
    \end{column}
  \end{columns}
\end{frame}

\subsection{Verzeichnisinhalte anzeigen}
\begin{frame}{Verzeichnisinhalte anzeigen}
  \begin{block}{Befehl}
    \co{\$ ls} (list) \\
    $\rightarrow$ zeigt Inhalt des aktuellen Verzeichnisses in Tabellenform farbig an
  \end{block}
  \begin{block}{Markierungen der Dateitypen}
    \begin{tabular}{lccc}
      Dateityp                  & Farbe         & Zeichen \co{-F} & Kennung \co{-l} \\
      \hline
      normale Datei             & schwarz       & keins & \co{-} \\
      ausführbare Datei         & grün          & * & \co{-} \\
      Verzeichnis               & blau          & / & \co{d} \\
      Link                      & cyan          & @ & \co{l} \\
    \end{tabular}
  \end{block}
\end{frame}


\begin{frame}{Optionen zum Anzeigen von Verzeichnisinhalten mit \co{ls}}
  \begin{block}{Allgemeine }
    \begin{tabular}{p{0.38\textwidth}p{0.6\textwidth}}
      \co{-a} oder \co{--all}           & auch versteckte Dateien anzeigen \\
      \co{-i} oder \co{--inode}         & eindeutige Dateinummer (Inode-Nr.)\\
      \co{-l} oder \co{--format=long}   & zusätzliche Informationen\\
      \co{--color=no}                   & keine Farbcodierung nach Dateityp \\
      \co{-p} oder \co{-F}              & Dateityp durch Sonderzeichen \\
      \co{-R} oder \co{--recursive}     & Unterverzeichnisse durchsuchen \\
    \end{tabular}
  \end{block}
  \begin{block}{Sortieroptionen}
    \begin{tabular}{ll}
      \co{-r} oder \co{--reverse}       & Sortierreihenfolge umkehren \\
      \co{-S} oder \co{--sort=size}     & nach Dateigröße  (groß $\rightarrow$ klein) \\
      \co{-t} oder \co{--sort=time}     & nach Zeit (neu $\rightarrow$ alt) \\
      \co{-x} oder \co{--sort=extension}       & nach Dateityp \\
    \end{tabular}
  \end{block}
  $\rightarrow$ Wichtig für die Prüfung
\end{frame}

\subsection{Verzeichnisse erstellen und löschen}
\begin{frame}{Verzeichnisse erstellen und löschen}
  \begin{block}{Verzeichnisse anlegen mit \co{mkdir}}
    \begin{itemize}
      \item ``make directory'': \co{mkdir}
      \item \co{\$ mkdir Verzeichnis}
      \item Option \co{-p}: übergeordnete Verzeichnisse mit erzeugen
    \end{itemize}
  \end{block}
  \begin{block}{Verzeichnisse löschen mit \co{rmdir}}
    \begin{itemize}
      \item ``remove directory'': \co{rmdir}
      \item Option \co{-p}: übergeordnete Verzeichnisse löschen
      \item Verzeichnisse dürfen keine Dateien enthalten
    \end{itemize}
  \end{block}
\end{frame}

\section{Suchmuster für Dateien}
\subsection{Einfache Suchmuster}
\begin{frame}{Suchmuster und Shell-Expansion}
%   \begin{block}{Grundlegendes}
    \begin{itemize}
      \item \emph{Expansion ist eine Leistung der Shell}
      \item erlauben ein einfaches Auswählen von vielen ähnlichen Dateien ohne explizite Eingabe
      \item Frage: Was geschieht, wenn eine Datei mit einem Minus ``\co{-}'' beginnt?
      \item Wenn die Expansion fehl schlägt: Ausdruck unverändert an das Programm übergeben
      \item \co{*}: Passt auf beliebig viele Zeichen (auch kein Zeichen)
      \item \co{?}: Passt auf genau ein beliebiges Zeichen
      \item beide passen auf alle Zeichen außer \co{/}
      \item versteckte Dateien müssen explizit ausgewählt werden (z.B.: \co{.*})
    \end{itemize}
%   \end{block}
\end{frame}

\subsection{Zeichenklassen}
\begin{frame}
  \begin{block}{Zeichenklassen}
    \begin{itemize}
      \item genau ein Zeichen, die mit \co{[} und \co{]} eingeschlossen sind
      \item Angabe direkt (\co{[abcdef]}) oder durch Bereiche (\co{[a-f]})
      \item Bereiche erstrecken sich über den ASCII-Tabelle
      \item Problem Umlaute: können nicht über \co{[a-z]} ausgewählt werden 
        $\rightarrow$ explizite Angabe
      \item Ausschluss von Zeichen mit \co{!}, z.B. \co{prog[!A-Z].cpp}
    \end{itemize}
  \end{block}
  \begin{block}{Beispiele}
    \begin{itemize}
      \item \co{[abc]-text.txt} passt auf die Dateien \co{a-text.txt}, \co{b-text.txt} und \co{c-text.txt}
    \end{itemize}
  \end{block}
\end{frame}

\subsection{Geschweite Klammern}
\begin{frame}{Geschweifte Klammern -- Das kartesische Produkt}
  \begin{block}{Das kartesische Produkt}
    \begin{itemize}
      \item Unabhängig von der Existenz als Dateinamen
      \item Generiert auf textuelle Basis Wörter
      \item Bereiche über ``\co{..}'' angebbar
      \item Bsp.: 
      \begin{tabular}{p{9.55em}p{0.55em}l}
        \co{\{rot,blau,gelb\}.txt}& $\rightarrow$ & \co{rot.txt blau.txt gelb.txt}\\
        \co{\{a,b\}\{1,2\}}        & $\rightarrow$ & \co{a1 a2 b1 b2} \\
        \co{\{a..d\}.txt}          & $\rightarrow$ & \co{a.txt b.txt c.txt d.txt}\\
        \co{\{a..f..2\}.txt}       & $\rightarrow$ & \co{a.txt c.txt e.txt}\\
      \end{tabular}
    \end{itemize}
  \end{block}
\end{frame}


\section{Arbeiten mit Dateien}
\subsection{Kopieren, Verschieben, Löschen}
\begin{frame}{Kopieren, Verschieben, Löschen}
  \begin{block}{Kopieren (``copy'')}
    \begin{itemize}
      \item Erzeugen einer neuen Datei als Kopie einer alten Datei
      \item \co{\$ cp original kopie} \\ 
          $\rightarrow$ erzeugt 1:1 Kopie vom Original
      \item \co{\$ cp original zielverzeichnis/ } \\ 
          $\rightarrow$ erzeugt 1:1 Kopie im Verzeichnis
      \item \co{\$ cp quelle1 quelle2 \dots ~zielverzeichnis/ } \\ 
          $\rightarrow$ erzeugt 1:1 Kopien der Dateien im Verzeichnis
    \end{itemize}
  \end{block}
\end{frame}

\begin{frame}{}
  \begin{block}{Verschieben und Umbenennen (``move'')}
    \begin{itemize}
      \item Befehl \co{mv} analog wie \co{cp}
      \item Verschieben einer Datei von einem Ort zu einem anderen
      \item im einfachsten Fall wird die Datei umbenannt
      \item Datei wird nur in neues Verzeichnis eingeordnet 
    \end{itemize}
  \end{block}
  \begin{block}{Löschen (``remove'')}
    \begin{itemize}
      \item \co{\$ rm datei}
        $\rightarrow$ löscht die Datei ``datei''
      \item \emph{Löschen ist endgültig!}
      \item Verzeichnis und Unterverzeichnisse mit Dateien löschen: \co{-rf}
      \item Frage: Was macht \co{rm -rf /}  ?
      \item Vorsichtiger Umgang mit Wildcards (\co{\textasteriskcentered}, \dots)!
    \end{itemize}
  \end{block}
\end{frame}
\begin{frame}{}
  \begin{block}{Allgemeine Arbeitsweise beim Kopieren, Verschieben,\dots}
    \begin{itemize}
      \item i.Allg. werden die Dateioperationen ohne Nachfrage ausgeführt\\
        $\rightarrow$ Dateien können ungefragt überschrieben werden
      \item Entscheidend sind die Berechtigungen in den jeweiligen Verzeichnissen 
        und nicht die der Dateien
    \end{itemize}
  \end{block}
  \begin{block}{Allgemeine Optionen für \co{cp}, \co{mv} und \co{rm}}
    \begin{tabular}{p{1.0em}p{0.2\textwidth}p{0.66\textwidth}}
%       \co{-b}   & (backup)      & Sicherungskopien ($\sim$) von existierenden Zieldateien \\
      \co{-f}   & (force)       & Aktion ohne Rückfrage erzwingen \\
      \co{-i}   & (interactive) & nachfragen für jede Aktion\\
      \co{-v}   & (verbose)     & ausführliche Ausgabe aller Aktionen \\
    \end{tabular}
  \end{block}
\end{frame}


\begin{frame}{Weitere Optionen}
  \begin{block}{Spezielle Optionen bei vorhandenen Dateien (\co{cp}, \co{mv})}
    \begin{tabular}{p{1.0em}p{0.19\textwidth}p{0.75\textwidth}}
      \co{-b}   & (backup)      & Sicherungskopien (\textasciitilde) von vorhandenen Zieldateien \\
      \co{-u}   &  (update)     & nur mit neueren Dateien überschreiben\\
    \end{tabular}
  \end{block}
  \begin{block}{Weitere spezielle Optionen}
    \begin{tabular}{p{1.0em}p{0.19\textwidth}p{0.66\textwidth}}
      \co{-p}   & (preserve)      & Dateiattribute behalten (cp) \\
      \co{-r}   & (recursive)     & rekursives Abarbeiten von Verzeichnissen (\co{cp}, \co{rm} ) \\
    \end{tabular}
  \end{block}
\end{frame}

\subsection{Verknüpfungen von Dateien}
\begin{frame}{Verknüpfungen zu Dateien}
  \begin{block}{Harte Verknüpfungen}
    \begin{itemize}
      \item \co{\$ ln datei hardlink}\\
        $\rightarrow$ ``datei'' und ``hardlink'' verweisen auf die gleiche Datei
        \item Verweisen auf die Inode-Nr. (Vgl. \co{ls -i})
        \item ``datei'' und ``hardlink'' sind ein und der selbe Name für die gleiche Datei
        \item Zweite Spalte von \co{ls -l} gibt Anzahl der Referenzen ($\ge 1$) an
        \item Beide Namen sind gleichwertig und ununterscheidbar
        \item Verknüpfung nur auf Dateien im selben Dateisystem möglich
    \end{itemize}
  \end{block}
\end{frame}

\begin{frame}{Verknüpfungen zu Dateien und Verzeichnisse}
  \begin{block}{Symbolische (weiche) Verknüpfungen}
    \begin{itemize}
      \item \co{\$ ln -s datei softlink}
      \item Verweist auf den Namen der Zieldatei
      \item Verknüpfung auf Dateien und Verzeichnisse möglich
    \end{itemize}
  \end{block}
  \begin{block}{Weitere Optionen für \co{ln}}
    Optionen \co{-b}, \co{-f}, \co{-i} und \co{-v} wie bei \co{cp}
  \end{block}
\end{frame}


\subsection{Inhalte von Dateien anzeigen lassen}
\begin{frame}%{Textdateien lesen}
  \begin{columns}
    \begin{column}{0.34\textwidth}
%         \begin{block}{Mehr lesen mit \co{more}}
        \begin{block}{Mögliche Programme}
        \begin{itemize}
%           \item Rückwärtsbewegung im Text nicht möglich
          \item  \co{more}:   unkomfortabel
          \item Weniger ist mehr:\\ \co{less}\\
          $\rightarrow$ besser bedienbar
        \end{itemize}
      \end{block}
%         \begin{block}{Weniger ist mehr -- \co{less}}
% %           \begin{itemize}
% %             \item 
%             Textnavigation auch mit Pfeiltasten
% %           \end{itemize}
%         \end{block}
%         \flushbottom
%         \begin{block}{}
%           \begin{tabular}{ll}
%           \end{tabular}
%         \end{block}
    \end{column}
    \begin{column}{0.71\textwidth}
        \begin{block}{Tastaturbefehle für \co{less}}
          \begin{tabular}{cl}
            \taste{$\hookleftarrow$} od. \taste{j}  & nächste Zeile (Pfeiltasten) \\
            \taste{ } od. \taste{f}   & nächste Seite \\
            \taste{y} od. \taste{k}  & vorherige Zeile \\
            \taste{b}   & Seite zurück \\
            \taste{h}   & Hilfetext \\
            \taste{q}   & beenden \\
            \taste{!}\co{befehl} \taste{$\hookleftarrow$}   & \co{befehl} ausführen \\
            \taste{v}   & Editor öffnen\\
            \taste{Strg} \taste{l}  & Bildschirm neuzeichnen\\
            \taste{p}\co{zahl}  \taste{$\hookleftarrow$}  & gehe zu \co{zahl}\% des Textes\\
            \taste{/}\co{wort} \taste{$\hookleftarrow$}   & sucht \co{wort} abwärts\\
            \taste{?}\co{wort} \taste{$\hookleftarrow$}   & aufwärts suchen\\
            \taste{n}   & abwärts weitersuchen \\
            \taste{N}   & aufwärts weitersuchen \\
          \end{tabular}
        \end{block}
    \end{column}
  \end{columns}
\end{frame}

\subsection{Dateien suchen und finden}
\begin{frame}{Suchen und Finden im Verzeichnisbaum}
  \begin{block}{Rekursives Suchen mit \co{find}}
    Allgemeiner Syntax:\\
    \co{find startverzeichnis kriterium aktion}
    \begin{itemize}
%       \item \co{find startverzeichnis kriterium aktion}
      \item alle als \co{startverzeichnis} angegebene Verzeichnisse werden rekursiv durchsucht
      \item \co{kriterium} beschreibt die Bedingungen, die erfüllt sein müssen
      \item \co{aktion} legt fest, was mit einem Treffer geschieht
    \end{itemize}
  \end{block}
\end{frame}
\begin{frame}{Suchkriterien}
        \begin{tabular}{lp{.8\textwidth}}
          \co{-name}        & Name muss auf das Muster passen\\
          \co{-iname}        & Groß-/Kleinschreibung ignorieren\\
          \co{-user}        & Eigentümer stimmt überein\\
          \co{-group}        & Gruppe\\
          \co{-type}        & bestimmter Dateityp\\
          \co{-size}        & Dateigröße in 512 B (c für Byte,k für kiB, Mindestwert mit Pluszeichen, Maximalwert mit Minuszeichen)\\
          \co{-atime}        & Zeit des letzten Zugriffs in Tagen\\
          \co{-amin}        & Zeit des letzten Zugriffs in Minuten\\
          \co{-mtime}        & Zeit der letzten Änderung\\
          \co{-ctime}        & Zeit der letzten Änderung am Inode\\
          \co{-perm}        & Zugriffsberechtigungen als Oktalzahl\\
          \co{-links}        & Anzahl Referenzen\\
          \co{-inum}        & Inode-Nummer\\
        \end{tabular}
\end{frame}

\begin{frame}{Mehrere Suchkriterien logisch verknüpfen}
       \begin{block}{Verknüpfen von Kriterien}
        \begin{itemize}
          \item mehrere Kriterien automatisch UND-Verknüpft
          \item logische Operatoren:
          \begin{description}
           \item[Und:]  \co{-a}
           \item[Oder:] \co{-o}
           \item[Nicht:] \co{!}
          \end{description}
          \item Zusammenfassen durch runde Klammern: \\
                \co{\textbackslash ( krit1 -o krit2 \textbackslash )}\\
		(Klammern vor Shell mit Rückstrichen maskiert)
        \end{itemize}
      \end{block}
\end{frame}


\begin{frame}{Funde verarbeiten}
  \begin{block}{Aktionen}
    Aktionen, die mit einem gefundenen Dateinamen ausgeführt werden\\
    \begin{tabular}{ll}
      \co{-print}     & Funde ausgeben\\
      \co{-exec}     & Befehl ausführen\\
      \co{-ok}     & nach Zustimmung Befehl ausführen\\
    \end{tabular}

  \end{block}
  \begin{block}{Dateinamen sammeln mit \co{xarg}}
      \co{\$ find . -user nobody | xarg -r rm -i}\\
      $\rightarrow$ \co{xarg} sammelt Namen und gibt sie dann an \co{rm} zusammen weiter
  \end{block}

\end{frame}

\begin{frame}{Schneller Suchen dank Datenbanken}
  \begin{block}{\co{locate}}
    \begin{itemize}
%       \item nicht standardmäßig installiert
      \item sucht in Datenbank nach passenden Einträgen\\
        $\rightarrow$ Erneuern der Datenbank mit \co{updatedb} (langsam)
        \begin{itemize}
          \item meist automatisch über Programme (\co{cron}) im Hintergrund
          \item Auswahl der Verzeichnisse über Konfigurationsdateien der Distribution
          (z.B. \co{/etc/updatedb.conf})
        \end{itemize}
      \item Sicherheitsproblem: updatedb speichert keine Berechtigungen
      \item Alternative \co{slocate} berücksichtig Zugriffsberechtigungen
      \item \co{\$ locate 'suchmuster'} \\
          \co{suchmuster} kann ein Teil innerhalb des kompletten Pfades sein oder 
          ein Suchmuster wie in der Shell (komplette Übereinstimmung)
    \end{itemize}
  \end{block}
  \begin{block}{\co{slocate}}
    Berücksichtig Eigentümer und Zugriffsberechtigungen beim Suchen
  \end{block}
\end{frame}
% 
% \begin{frame}{}
%   \begin{block}{}
%     \begin{itemize}
%       \item 
%     \end{itemize}
%   \end{block}
% \end{frame}
% 


\begin{frame}{Prüfungsziele I}
  \begin{alertblock}{2.3 Verzeichnisse und Protokolldateien nutzen}
    \begin{description}
      \item[Gewicht]  2
      \item[Beschreibung] Navigieren in Home- und Systemverzeichnissen und Protokolldateien an versch. Orten
    \end{description}
       Wissensgebiete\\ 
        \begin{itemize}
          \item Dateien, Verzeichnisse
          \item Versteckte Dateien und Verzeichnisse
          \item Benutzerverzeichnisse
          \item Absolute und relative Pfade
          \item \co{ls}, \co{cd}, \co{HOME} und \co{-}
          \item \co{.} und \co{..}
        \end{itemize}
  \end{alertblock}
\end{frame}

\begin{frame}{Prüfungsziele II}
  \begin{alertblock}{2.4 Erstellen, Verschieben und Löschen von Dateien}
    \begin{description}
      \item[Gewicht]  2
      \item[Beschreibung] Erstellen, Verschieben und Löschen von Dateien und Verzeichnissen im eigenen Verzeichnis
    \end{description}
       Wissensgebiete\\ 
        \begin{itemize}
          \item Dateien, Verzeichnisse
          \item Groß- und Kleinschreibung
          \item einfaches Globbing und Quoting
          \item \co{mv}, \co{cp}, \co{rm}, \co{touch}
          \item \co{mkdir}, \co{rmdir}
        \end{itemize}
  \end{alertblock}
\end{frame}
%%%%%%%%%%%%%%%%%%%%%%%%%% next time...
\begin{frame}[plain]
  \begin{alertblock}{Nächste Vorlesung}
    \textbf{Termin:} In 15 Minuten\\
    \textbf{Thema:} Kapitel 5 - Suchen und Finden \\
    \textbf{Lehrbuchkapitel:} 
    \begin{itemize}
      \item LXES 7 Reguläre Ausdrücke
      \item LXES 8 Standardkanäle und Filterkommandos
    \end{itemize}
  \end{alertblock}
\end{frame}

\materialframe
\versionframe

\end{document}
