\documentclass[aspectratio=43]{beamer}
\usepackage{../../resources/agdsncourse} 

\title[Linux Essentials  - Kapitel 5 - Suchen und Finden]{Linux Essentials}
\subtitle{Kapitel 5 - Suchen und Finden}
\author{Hagen Eckert}
\date{18. Oktober 2015}


\begin{document}

% Start Folie
\logoframe

%Title
\frame{\titlepage}

%Gliederung
\setcounter{tocdepth}{1}
\section[Gliederung]{}
\frame{\tableofcontents}

% Inhalt
%%%%%%%%%%%%%%%
%%%%%%%%%%%%%%%%%%%%%%%%%%%%
\section{Reguläre Ausdrücke}
\subsection{Syntax}
\begin{frame} 
	\begin{block}{Syntax} 
	\begin{itemize}
	\item \textbf{Literale} normaler Text
	\newline  das Zeichen m steht eben für ein m 
	\item \textbf{Metazeichen} Zeichen besonderer Bedeutung, 
	\newline ? oder .
	\item \textbf{Positonszeiger} für die genaue Position, an der ein Zeichen gesucht wird
	\newline  $\hat{\;}\;$ \$
	\item \textbf{Zeichenfolge} Menge von Zeichen, die gefunden werden soll
	\newline  [abcde] [a-e] [0­-9] [$\;\hat{}\;$Negation]
	\item \textbf{Quantifizierer} modifizieren die Anzahl an Zeichen, die gefunden wird.
	\newline  ? . + * \{n,m\}
	\end{itemize}
	\end{block}


\end{frame}


\subsection{grep}
\begin{frame} 
	\begin{exampleblock}{Beispiel I}
	\co{egrep \'\textbackslash$<$[0-­9]\{5\}\textbackslash$>$\'}
	\end{exampleblock}

	\begin{block}{Postleitzahlen finden} 
Dieser Ausdruck sucht nach fünfstelligen Zahlen, wie zum Beispiel Postleitzahlen. \\
\co{[0-­9]} findet eine Ziffer zwischen 0 und 9. \\
Durch den Quantifizierer \co{\{5\}} dahinter muss etwas 5x hintereinander gefunden werden. \\
Die Wortgrenzen \co{\textbackslash$<$} und \co{\textbackslash$>$} sorgen schließlich dafür, dass nichts herum stehen darf.
	\end{block}	
\end{frame}

\begin{frame} 
	\begin{exampleblock}{Beispiel II}
	\co{egrep \'[$\hat{\;}$\textbackslash b]\textbackslash @[$\hat{\;}$\textbackslash b]\'}
	\end{exampleblock}

	\begin{block}{E­-Mail­-Adressen finden} 
Ausdruck zum Aufspüren von E­-Mail­-Adressen. \\
\co{\textbackslash b} bedeutet Wortanfang oder ­-ende. \\
Durch das $\hat{\;}$ zu Beginn der eckigen Klammern wird es verneint. 
\\So wird von obigem Ausdruck alles gefunden, was aus einem @ besteht, das links und rechts von mindestens einem Wortzeichen umrahmt wird. 
	\end{block}	
\end{frame}

\begin{frame} 
	\begin{exampleblock}{Beispiel III}
	\co{egrep \'([0-­9]\{3\}\textbackslash .)\{3\}[0­-9]\{3\}\'}
	\end{exampleblock}

	\begin{block}{IP­-Adressen finden} 
Ausdruck für einen einfachen IP-­Adressen-­Finder.\\
Es wird nach vier dreistelligen Zahlen die durch Punkte getrennt sind gesucht.
	\end{block}	
\end{frame}

\begin{frame} 
	\begin{exampleblock}{Beispiel IV}
	{\fontsize{5.5}{5.5} \selectfont\co{egrep \'\textbackslash$<$((([01]?[0­-9]\{1,2\})|(2[0­-4][0­-9])|(25[0-­5]))\textbackslash.)\{3\}(([01]?[0-­9]\{1,2\})|(2[0-­4][0-­9])|(25[0­-5]))\textbackslash$>$\'}}
	\end{exampleblock}

	\begin{block}{IP­-Adressen finden (jetzt echt)} 
Die Optimierung: Es wurden\textbackslash$<$und \textbackslash$>$ ergänzt, damit beispielsweise ag1234.122.33.4.012sd9 nicht gefunden wird. Anschließend wurde das [0­-9]\{3\} durch eine präzisere Variante ersetzt. Sie besteht aus drei Alternativen: [01]?[0-­9]\{1,2\} findet alle ein­ und zweistelligen Zahlen und dreistellige, die mit 1 beginnen. Jeweils eingeschlossen sind auch die Entsprechungen mit führenden Nullen (auffüllend bis drei Stellen). Es sind jetzt also alle Zahlen von 0 bis 199 abgedeckt. 2[0­-4]\{0­-9\} findet alle Zahlen von 200 bis 249. 25[0-­5] alle von 250 bis 255. Damit ist die Suche komplett.
	\end{block}	
\end{frame}

\begin{frame} 
	\begin{exampleblock}{Beispiel V}
	\co{egrep -i \'\textbackslash b([a-z]+) +\textbackslash 1\textbackslash b\'}\\
	\end{exampleblock}

	\begin{block}{Doppelte Worte finden} 
Mit dem Schalter \co{-i} sucht grep per 'ignore case'.\\
grep merkt sich den ersten gefundenen Ausdruck in der Klammer.\\
Nach der Klammer kommt  mindestens ein Leerzeichen ( +) und dann die Metasequenz die auf das gefundene Wort passt \textbackslash 1 danach die Wortgrenze \textbackslash b.
	\end{block}	
\end{frame}

\subsection{sed}
\begin{frame} 

	\begin{block}{sed} 
	\begin{itemize}
	\item ein stream editor
	\item eigentlich wie vi nur ohne Interaktion
	\item es werden Zeile für Zeile Anweisungen am Text ausgeführt 
	\end{itemize}
	\end{block}

\end{frame}

\section{Standardkanäle und Filterkomandos}
\subsection{Umleitung}
\begin{frame} 

	\begin{block}{Kanäle} 
	\begin{itemize}
	\item 0 Standard-Eingabe \co{stdin}
	\item 1 Standard-Ausgabe \co{stdout}
	\item 2 Standard-Fehlerausgabe \co{stderr}
	\end{itemize}
	\end{block}

	\begin{exampleblock}{Umleitung} 
	\begin{itemize}
	\item \co{>} \co{stdout} umleiten  in eine Datei
	\item \co{>>} \co{stdout} umleiten  in eine Datei aber anhängen
	\item \co{1>} mit einer Zahl wird ein bestimmter Kanal angesprochen
	\item \co{2>\&1}  \co{stderr} nach \co{stdout} umleiten
	\item \co{|} Pipe verbindet \co{stdout} eines Prozesses mit dem \co{stdin} eines anderen
	\item \co{tee} gibt sein \co{stdin} als Datei und als \co{stdout} wieder
	\end{itemize}
	\end{exampleblock}



\end{frame}

\subsection{Befehlsammlung}
\begin{frame} 
	\begin{block}{\co{cat}} 
        Dateien aneinanderhängen und zur Standardausgabe schreiben
	\end{block}
	\begin{block}{USAGE} 
        \co{cat [OPTION] [DATEI]...}
	\end{block}
	\begin{exampleblock}{Optionen} 
	\begin{itemize}
	\item \co{-b, --number-nonblank} Nummeriere alle Zeilen, die nicht NULL sind
	\item \co{-n, --number} Nummeriere alle Zeilen
	\item \co{-t} zeige alle \taste{$\leftrightharpoons\quad$}  als $\,\hat{}$\,I
	\item \co{-s, --squeeze-blank} nie mehr als eine Leerzeile in der Ausgabe
	\end{itemize}
	\end{exampleblock}
\end{frame}

\begin{frame} 
	\begin{block}{\co{tac}} 
        Dateien umkehren und aneinandergehängt ausgeben
	\end{block}
	\begin{block}{USAGE} 
        \co{tac [OPTION] [DATEI]...}
	\end{block}
	\begin{exampleblock}{Optionen} 
	\begin{itemize}
	\item \co{ -s, --separator=ZEICHENKETTE}  
	\newline \co{ZEICHENKETTE} als Trennzeichen statt Zeilenumbruch benutzen
	\end{itemize}
	\end{exampleblock}
\end{frame}

\begin{frame} 
	\begin{block}{\co{head}} 
        Den ersten Teil von Dateien ausgeben
	\end{block}
	\begin{block}{USAGE} 
        \co{head [OPTION] [DATEI]...}
	\end{block}
	\begin{exampleblock}{Optionen} 
	\begin{itemize}
	\item \co{-c, --bytes=[-]N} \newline Die ersten \co{N} Bytes jeder Datei ausgeben; mit führendem Minus, alle außer den letzten \co{N} Bytes jeder Datei	
	\item \co{-n, --lines=[-]N}\newline Wie \co{-c} nur mit Zeilenanzahl \co{N}
	\end{itemize}
	
	\end{exampleblock}
\end{frame}

\begin{frame} 
	\begin{block}{\co{tail}} 
        Den letzten Teil von Dateien ausgeben
	\end{block}
	\begin{block}{USAGE} 
        \co{tail [OPTION] [DATEI]...}
	\end{block}
	\begin{exampleblock}{Optionen} 
	\begin{itemize}
	\item \co{ -f, --follow[=\{name\}]} \newline Angefügte Daten ausgeben,  wahrend  die  Datei  wächst	
	\item \co{--retry}\newline  Weiterhin  versuchen, eine Datei zu öffnen, selbst wenn sie beim
              Start oder später nicht verfügbar ist
	\end{itemize}
	
	\end{exampleblock}
\end{frame}


\begin{frame} 
	\begin{block}{\co{hexdump}} 
        		ASCII, decimal, hexadecimal, octal dump
	\end{block}
	\begin{block}{USAGE} 
        		\co{hexdump [-bcCdovx] [-e formatstring]  [-f formatfile]}\\
        		\hspace{1.4cm}\co{ [-n length] [-s skip] file ...}
	
	\end{block}

	\begin{exampleblock}{Optionen} 
	\begin{itemize}
	\item \co{-b} One-byte octal \co{-o} Two-byte octal 
	\item \co{-c} One-byte character 
	\item \co{-C} Canonical hex+ASCII  
	\item \co{-d} Two-byte decimal 
	\item \co{-x} Two-byte hexadecimal 
	\end{itemize}
	
	\end{exampleblock}



\end{frame}

\begin{frame} 
	\begin{block}{\co{tr}} 
        Zeichen umwandeln oder löschen
	\end{block}
	\begin{block}{USAGE} 
        \co{tr [OPTION]... MENGE1 [MENGE2]}
	\end{block}
	\begin{exampleblock}{Optionen} 
	\begin{itemize}
	\item \co{-d, --delete} \newline Zeichen aus \co{MENGE1} loschen, nicht umwandeln
	\item \co{-s, --squeeze-repeats} \newline Jede Eingabe-Sequenz mit sich wiederholenden Zeichen aus  \text{MENGE1}
              durch ein einziges Vorkommen dieses Zeichens ersetzen
	\end{itemize}
	
	\end{exampleblock}
\end{frame}

\begin{frame} 
	\begin{block}{\co{expand}} 
        Tabulatoren in Leerzeichen umwandeln
	\end{block}
	\begin{block}{USAGE} 
        \co{expand [OPTION]... [DATEI]...}
	\end{block}
	\begin{exampleblock}{Optionen} 
	\begin{itemize}
		\item \co{-t, --tabs=ZAHL} \newline Tabulatoren entsprechen \co{ZAHL} Zeichen, nicht 8
	\item \co{-t, --tabs=LISTE} \newline Durch  Komma  getrennte LISTE von expliziten Tabulatorpositionen verwenden
	\item \co{-i, --initial} \newline Tabulatoren nicht nach  Nicht-Freiraumzeichen umwandeln
	\end{itemize}
	
	\end{exampleblock}
\end{frame}

\begin{frame} 
	\begin{block}{\co{unexpand}} 
        Leerzeichen in Tabulatoren umwandeln
	\end{block}
	\begin{block}{USAGE} 
        \co{unexpand [OPTION]... [DATEI]...}
	\end{block}
	\begin{exampleblock}{Optionen} 
	\begin{itemize}
	\item \co{-a, --all} \newline Alle Leerzeichen umwandeln, statt nur der führenden
	\item \co{--first-only} \newline Nur führende Leerzeichen umwandeln (überschreibt -a)
	\item \co{-t, --tabs=N} \newline Tabulatoren haben \co{N} Zeichen, statt 8 (schaltet -a ein).
	\item \co{-t, --tabs=LISTE} \newline Liste von Tabulatorpositionen  verwenden (schaltet -a ein).
	\end{itemize}
	
	\end{exampleblock}
\end{frame}

\begin{frame} 
	\begin{block}{\co{fmt}} 
         Einfacher optimaler Textformatierer
	\end{block}
	\begin{block}{USAGE} 
          \co{fmt [-BREITE] [OPTION]... [DATEI]...}
	\end{block}
	\begin{exampleblock}{Optionen} 
	\begin{itemize}
	\item \co{-u, --uniform-spacing} \newline Ein Leerzeichen zwischen Wörtern, zwei nach Sätzen
	\item \co{-s, --split-only} \newline Lange Zeilen aufteilen, aber nicht auffüllen
	\end{itemize}
	
	\end{exampleblock}
\end{frame}

\begin{frame} 
	\begin{block}{\co{pr}} 
        Textdateien fur Druckausgabe umwandeln
	\end{block}
	\begin{block}{USAGE} 
        \co{pr [OPTION]... [DATEI]...}
	\end{block}
	\begin{exampleblock}{Optionen} 
	\begin{itemize}
	\item \co{-h --header=KOPF} \newline   \co{KOPF}   als   zentrierten  Seitenkopf  anstelle  des  Dateinamens   benutzen. 
	\item \co{-l, --length=SEITENLAENGE} \newline Seitenlange auf \co{SEITENLAENGE}  (66)  Zeilen  setzen  
	\item \co{-o, --indent=MARGIN} \newline Versetzt  jede  Zeile  um  \co{MARGIN}  (0) Leerzeichen
	\end{itemize}
	
	\end{exampleblock}
\end{frame}

\begin{frame} 
	\begin{block}{\co{nl}} 
        Zeilen von Dateien nummerieren
	\end{block}
	\begin{block}{USAGE} 
        \co{nl [OPTION]... [DATEI]...}
	\end{block}
	\begin{exampleblock}{Optionen} 
	\begin{itemize}
	\item \co{-h / -b / -f} \newline Setzten von Header/Body/Footer auf \co{a, t} oder \co{n}
	\item \co{a} Nummeriere alle Zeilen
	\item \co{t} Nummeriere nur nicht-leere Zeilen
	\item \co{n} Nummeriere nicht
	\item \co{-w, --number-width=ANZAHL\newline   ANZAHL} Spalten für Zeilennummern benutzen
	
	\end{itemize}
	
	\end{exampleblock}
\end{frame}

\begin{frame} 
	\begin{block}{\co{wc}} 
         Anzahl der Zeilen, Wörter und Bytes für jede Datei ausgeben
	\end{block}
	\begin{block}{USAGE} 
        \co{wc [OPTION]... [DATEI]...}
	\end{block}
	\begin{exampleblock}{Optionen} 
	\begin{itemize}
	\item \co{-c, --bytes}  Anzahl der Bytes ausgeben
	\item \co{-m, --chars} Anzahl der Zeichen ausgeben
	\item \co{-l, --lines} Anzahl der Zeilen ausgeben
	
	\end{itemize}
	
	\end{exampleblock}
\end{frame}


\begin{frame} 
	\begin{block}{\co{sort}} 
        Zeilen von Textdateien sortieren
	\end{block}
	\begin{block}{USAGE} 
        \co{sort [OPTION]... [DATEI]...}
	\end{block}
	\begin{exampleblock}{Optionen} 
	\begin{itemize}
	\item \co{-b, --ignore-leading-blanks} \newline Führende Leerzeichen ignorieren
	\item \co{-f, --ignore-case} Klein- als Großbuchstaben behandeln
	\item \co{-i, --ignore-nonprinting} \newline Nur druckbare Zeichen berücksichtigen
	\item \co{-r, --reverse} Das Ergebnis der Vergleiche umkehren
	\item \co{-n, --numeric-sort} \newline Numerischen Werts der Zeichenkette vergleichen
	\end{itemize}
	\end{exampleblock}
\end{frame}

\begin{frame} 
	\begin{block}{\co{uniq}} 
         Doppelte aufeinanderfolgende Zeilen berichten oder entfernen
	\end{block}
	\begin{block}{USAGE} 
        \co{uniq [OPTION]... [EINGABE [AUSGABE]]}
	\end{block}
	\begin{exampleblock}{Optionen} 
	\vspace{-0.3cm}
	\begin{itemize}
	\item \co{-c, --count} \newline Den Zeilen die Häufigkeit des Vorkommens voranstellen
	\item \co{-s, --skip-chars=N}   Die ersten \co{N} Zeichen nicht vergleichen
	\item \co{-i, --ignore-case} Klein- als Großbuchstaben behandeln
	\item \co{-f, --skip-fields=N} Die ersten \co{N} Felder nicht vergleichen
	\item \co{-s, --skip-chars=N} Die ersten \co{N} Zeichen nicht vergleichen
	\item \co{-u, --unique} Nur einmal auftretende Zeilen ausgeben	
	\end{itemize}
	\end{exampleblock}
\end{frame}

\begin{frame} 
	\begin{block}{\co{cut}} 
         Teile jeder Zeile aus Dateien ausgeben
	\end{block}
	\begin{block}{USAGE} 
        \co{cut OPTION... [DATEI]...}
	\end{block}
	\begin{exampleblock}{Optionen} 
	\begin{itemize}
	\item \co{-b, --bytes=LISTE} Nur diese Bytes auswählen
	\item \co{-c, --characters=LISTE} Nur diese Zeichen auswählen
	\item \co{-d, --delimiter=TRENNER}\newline  \co{TRENNER} anstelle des Tabulators  benutzen
	\item \co{-f, --fields=LISTE} Nur  diese  Felder  auswählen; außerdem jede Zeile ausgeben, die kein Trennzeichen enthält,  es  sei  denn,  die  Option  \co{-s}  ist angegeben
	\end{itemize}
	\end{exampleblock}
\end{frame}

\begin{frame} 
	\begin{block}{\co{join}} 
        Zeilen von zwei Dateien über ein gemeinsames Feld verbinden
	\end{block}
	\begin{block}{USAGE} 
        \co{join [OPTION]... DATEI1 DATEI2}
	\end{block}
	\begin{exampleblock}{Optionen} 
	\begin{itemize}
	\item \co{-j1 FELDNUM} Verknüpft die Dateien nach \co{FELDNUM} in \co{DATEI1}
	\item \co{-j2 FELDNUM} Verknüpft die Dateien nach \co{FELDNUM} in \co{DATEI2}
	\item \co{-j FELDNUM} \newline Verknüpft die Dateien nach \co{FELDNUM} in beiden Dateien
	\item \co{-t ZEICHEN}  \newline \co{ZEICHEN} als Trennzeichen für Ein- und Ausgabefelder nehmen
	\end{itemize}
	\end{exampleblock}
\end{frame}

\begin{frame} 
	\begin{block}{\co{paste}} 
        Zeilen von Dateien zusammenfügen
	\end{block}
	\begin{block}{USAGE} 
        \co{paste [OPTION]... [DATEI]..}
	\end{block}
	\begin{exampleblock}{Optionen} 
	\begin{itemize}
	\item \co{-d, --delimiters=LISTE} \newline Zeichen aus \co{LISTE} statt Tabulatoren benutzen
	\item \co{-s, --serial} \newline Eine Datei nach der anderen statt parallel ausgeben
	\end{itemize}
	\end{exampleblock}
\end{frame}


\begin{frame}[plain]
\begin{alertblock}{Nächste Vorlesung}
\textbf{Termin:} nach der Mittagspause\\
\textbf{Thema:} Hands-On\\
\end{alertblock}
\end{frame}

\materialframe

\end{document}
