\documentclass[aspectratio=43]{beamer}
\usepackage{../../resources/agdsncourse} 

\title[Linux Essentials  - Kapitel 1 - Einführung]{Linux Essentials}
\subtitle{Kapitel 1 - Einführung}
\author{Hagen Eckert}
\date{17. Oktober 2015}


\begin{document}

% Start Folie
\logoframe

%Title
\frame{\titlepage}

%Gliederung
\setcounter{tocdepth}{1}
\section[Gliederung]{}
\frame{\tableofcontents}


%%%%%%%%%%%%%%%
\section{Organisatorisches}
\subsection{Ziele \& Vorlesende}
\begin{frame} 

	\begin{block}{Ziele} 
	\begin{itemize}
	\item Grundwissen zur kompetenten Benutzung von Linux
	\item Einblicke in freie Software
	\item Aufbau und Zusammenspiel der Komponenten von Linux
	\item Grundlagen zur Benutzung der Kommandozeile
	\item Einführung in die Systemadministration
	\end{itemize}
	\end{block}
	
	\begin{block}{Vorlesende} 
	\begin{itemize}
	\item Alexander Köhler (HZDR)
	\item Hagen Eckert (TUD)
	\end{itemize}
	\end{block}

\end{frame}

\subsection{LPI}
\begin{frame} 

	\begin{block}{LPI} 
	\begin{itemize}
	\item LPI $\rightarrow$ \textbf{L}inux \textbf{P}rofessional \textbf{I}nstitute
	\item Gemeinnützige Organisation aus Kanada
	\item Betreibt das weltweit erfolgreichste Zertifizierungsprogramm der Linux-Gemeinschaft
	\item Distributionsunabhängig / International
	\end{itemize}
	\end{block}
	
	\begin{block}{LPI AAP} 
	\begin{itemize}
	\item Die AG DSN ist \textbf{A}pproved \textbf{A}cadamic \textbf{P}artner
	\item Eingetragenes Prüfungscenter für das Essential-Level
	\end{itemize}
	\end{block}
\end{frame}

\subsection{Veranstaltung}
\begin{frame} 
	\begin{block}{Aufbau} 
	\begin{itemize}
	\item 2 Wochenenden mit je 5 Themen + 1 Hands-On
	\item Konsultation + Prüfung
	\end{itemize}
	\end{block}
	
	\begin{block}{Zertifikat} 
	\begin{itemize}
	\item Computer basierte Prüfung mit sofortige Rückmeldung
	\item Kosten für die Zertifizierung sind 50 EUR (AG DSN 20 EUR)
	\end{itemize}
	\end{block}
	
	\begin{block}{Prüfung} 
	\begin{itemize}
	\item Alternativ kostenlose schriftliche Prüfung
	\item Ähnlicher Aufbau und Schwierigkeit wie die Zertifikatsprüfung
	\end{itemize}
	\end{block}
\end{frame}

\subsection{Partner}

\begin{frame}
	\begin{block}{Helmholtz-Zentrum Dresden-Rossendorf} 
	\begin{itemize}
	\item Institut für Strahlenphysik  
	\item Prof. Dr. Ulrich Schramm
	\item Grundlagenforschung auf dem Gebiet der Beschleuniger-, Kern-, Hadronen- und Laserphysik
	\end{itemize}
	\end{block}
	
\end{frame}

\begin{frame}
	\begin{block}{IfWW, TU Dresden} 
	\begin{itemize}
	\item Lehrstuhl für Materialwissenschaft und Nanotechnik 
	\item Prof. Dr. Cuniberti
	\item Konzepte, quantitative Methoden und Experimente um Phänomene im Nanometerbereich zu verstehen und zu beherrschen
	\end{itemize}
	\end{block}
		\begin{block}{} 
	\begin{itemize}
	\item Für die Lösung vieler Fragen werden Hochleistungsrechner eingesetzt  $\rightarrow$ 488 der Top 500 Hochleistungsrechner laufen unter Linux (Juni 2015, 10 Unix, 1 Windows) 
	% top500.org
	\end{itemize}
	\end{block}
\end{frame}



%%%%%%%%%%%%%%%%%
\section{Der Computer}
\subsection{Geschichte}
\begin{frame} 

	\begin{block}{Geschichte} 
	\begin{itemize}
	\item Beginn schwer abzugrenzen
	\item Antike Rechenhilfen und mathematische Methoden bilden dir Grundlage der Entwicklung hin zum modernen Computer
	\item Im 17. Jahrhundert kommen erste mechanische Rechner auf
	\item Das binäre Zahlensystem wird 1703 von Leibniz entwickelt
	\item Lochkarten erlauben das \textit{programmieren}
	\newline (1890 US-Volkszählung)
	\item 1941 ist die Zuse Z3 die erste Maschine die beliebige Algorithmen automatisch ausführen konnte.
	\end{itemize}
	\end{block}


\end{frame}

\subsection{Hardware}
\begin{frame} 

	\begin{block}{Prozessor} 
	\begin{itemize}
	\item \textbf{C}entral \textbf{P}rocessing \textbf{U}nit
	\item Besteht aus:
	\begin{itemize}
	 \item Arimethrisch-logische Einheit (ALU)
	 \item Steuerwerk
	 \item Register
	 \item Speichermanager (MMU)
	
	\end{itemize}
	\item Verschiedene Architekturen wie X86-64, ARM
	\end{itemize}
	\end{block}

	\begin{block}{RAM} 
	\begin{itemize}
	\item \textbf{R}andom-\textbf{A}ccess \textbf{M}emory
	\item Programmcode und zu verarbeitende Daten werden hier gespeichert
	\end{itemize}
	\end{block}
\end{frame}

\begin{frame} 

	\begin{block}{Coprozessor} 
	\begin{itemize}
	\item Sollen die CPU von verschiedenen Aufgaben entlasten
	\item Effizienter durch Spezialisierung 
	\item Grafikkarten, Soundkarten, IO-Erweiterungen, ...
	\item auch allgemein gehaltene Prozessoren möglich
	\end{itemize}
	\end{block}

	\begin{block}{Mainboard} 
	\begin{itemize}
	\item Bildet die Infrastruktur für die restliche Hardware
	\item Stellt Datenverbindungen zwischen den Komponenten her
	\item Verteilt Strom
	\item Besitzt eigenen Chips insbesondere für IO-Aufgaben
	\end{itemize}
	\end{block}
\end{frame}

\begin{frame} 

	\begin{block}{Speicher} 
	\begin{itemize}
	\item Dauerhafter Datenspeicher
	\item Für hohe Kapazität immer noch Festplatten am wichtigsten
	\item Durch die mechanischen Teile jedoch einige Nachteile
	\item Solid State Disk werden immer wichtiger, insbesondere wenn schnelle Übertragungs - und Zugriffsgeschwindigkeiten nötig sind
	\item Dennoch um Größenordnungen langsamer als der RAM 
	\begin{itemize}
	 \item[RAM] $6.0\cdot10^{-8}$ s
	\item[SSD] $2.5 \cdot10^{-4}$ s
	\item[HD] $1.0\cdot10^{-2}$ s
	\end{itemize}
	\end{itemize}
	\end{block}

\end{frame}

\begin{frame} 

	\begin{block}{Optische Laufwerke} 
	\begin{itemize}
	\item CD, DVD, Blu-ray
	\item Daten werden mit Laser gelesen $\rightarrow$ optisch
	\item Verlieren stark an Bedeutung durch USB-Sticks und \\Breitband Internet
	\end{itemize}
	\end{block}

	\begin{block}{Netzwerk} 
	\begin{itemize}
	\item Verschiedene Schnittstellen möglich
	\item Von 10BASE BNC bis 10Gbit/s Glasfaser möglich
	\item Meistens ist Gigabit-Ethernet über Kupferadern oder verschiedene WLAN-Varianten anzutreffen
	\item Glasfaser meist nur für Backbones und Server verwendet
	\end{itemize}
	\end{block}
\end{frame}

\subsection{Betriebssysteme}

% Source for market share StatCounter
% http://www.statista.com/statistics/268237/global-market-share-held-by-operating-systems-since-2009/

\begin{frame} 

	\begin{block}{Windows  78,93\% \footnote{Prozente von July 2015 StatCounter (17,2 Milliarden Seitenaufrufe)} }
	\begin{itemize}
	\item 1985 als grafischer Aufsatz für MS-DOS gestartet
	\item Seit Windows NT (1993) eigenständiges Betriebssysteme
	\item Mit XP (2001) auch im \textit{privaten} Bereich
	\item Seit 2010 auch mobiles System auf NT-Basis
	
	
	\end{itemize}
	\end{block}
	
	\begin{block}{OS X 7,79\% (iOS 6,03\%)} 
	\begin{itemize}
	\item Auf Darwin basierende Distribution (XNU $\rightarrow$ hybrider Kernel)
	\item Integration von GNU Programmen (GNU/XNU)
	\item Unix $\rightarrow$ BSD $\rightarrow$ NeXTStep $\rightarrow$ Darwin
	\item Seit 10.5 echtes UNIX (Open Group)
	\end{itemize}
	\end{block}

\end{frame}

\begin{frame} 

	\begin{block}{Linux 1,62\% (Android 1,62\%)} 
	\begin{itemize}
	\item Linux stellt eigentlich nur ein Betriebssystems Kern dar
	\item Dient häufig als Sammelbegriff für das gesamte System
	\newline Besser wäre: GNU/Linux
	\item Entwicklung geht auf Linus Torvalds zurück
	\item 1991 als Hobby Projekt begonnen
	\item Unix $\rightarrow$ Minix $\rightarrow$ Linux
	\item Kombiniert mit den GNU Programmen entstand ein komplettes Betriebssystem 
	\end{itemize}
	\end{block}

\end{frame}

\begin{frame} 

	\begin{block}{Linux} 
	\begin{itemize}
	\item Linux 2.0 wichtiger Meilstein
	\newline $\rightarrow$  Mehrkernprozessoren, dynamisches Laden von Modulen
	\item Ständige Weiterentwicklung bis heute
	\item Portiert auf fast alles was \textit{rechnen} kann
	\item Im allgemeinem PC Markt nie besonders erfolgreich
	\newline Erfolge eher im Bereich Embedded Systeme, Mobile Anwendungen, Server, wissenschaftliche Anwendung
	\end{itemize}
	\end{block}

\end{frame}

\section{Software}
\subsection{Distributionen}
\begin{frame} 

	\begin{block}{Distributionen} 
	\begin{itemize}
	\item Stellen komplette Softwareumgebungen bereit
	\item Kommandozeilenwerkzeuge, Arbeitsumgebung, Officeprogramme, Spiele, ...
	\item Große Vielfalt
	\newline $\rightarrow$ unterschiedliche Bedürfnisse und Philosophien 
	\end{itemize}
	\end{block}

	\begin{block}{Red Hat} 
	\begin{itemize}
	\item Gegründet 1993, aufkauf der Red Hat Distribution 1995
	\item Wohl größte Firma auf Basis von Linux und Open Source
	\item Verdient ihr Geld mit Support und speziellen Updateservices
	\item Eigene \textit{Testumgebung} Fedora
	\item Nachbauten wie CentOS
	\end{itemize}
	\end{block}

	
\end{frame}

\begin{frame} 

	\begin{block}{SUSE} 
	\begin{itemize}
	\item Basiert auf der ersten Linux-Disto SLS
	\item Bedeutend im deutschsprachigen Raum
	\item Wie Red Hat spezielle Enterprise Edition
	\item openSUSE für Endkunden
	\item Besonderheit ist YaST
	\end{itemize}
	\end{block}

	\begin{block}{Debian} 
	\begin{itemize}
	\item Ohne Firma im Hintergrund
	\item Begonnen 1993 mittlerweile  über 43.000 Pakete
	\item Strenge Regeln was freie Software ist
	\item Dient häufig als Grundlage anderer Projekte ($>50$ direkt + Ubuntu) 
	\end{itemize}
	\end{block}

	
\end{frame}

\begin{frame} 

	\begin{block}{Ubuntu} 
	\begin{itemize}
	\item Basiert auf Debian unstable
	\item Kürzere Erscheinungszyklen als Debian
	\item Mit den LTS-Versionen soll Konkurrenz zu etablierten Server Systemen geschaffen werden
	\item Unterstützt durch Canonical Ltd.
	\end{itemize}
	\end{block}

	\begin{block}{Weiter Distributionen} 
	\begin{itemize}
	\item Gentoo Linux $\rightarrow$ Quellcode
	\item Archlinux
	\item \textit{Android}
	\end{itemize}
	\end{block}

	
\end{frame}
\subsection{Anwendungsprogramme}
\begin{frame} 

	\begin{block}{Büroprogramme} 
	\begin{itemize}
	\item OpenOffice.org 
	\begin{itemize}
	 \item Schon viele Jahre bedeutendste freie Büropacket
	 \item Über SUN und Oracle bei der Apache Software Foundation gelandet
	\end{itemize}
	\item LibreOffice
	\begin{itemize}
	 \item Abspaltung von OpenOffice
	 \item Gilt als innovativer und aufgeräumter Nachfolger
	\end{itemize}


	\end{itemize}
	\end{block}

	\begin{block}{Internet} 
	\begin{itemize}
	\item Firefox
	\item Thunderbird
	\item Chromium
	\end{itemize}
	\end{block}

	
\end{frame}

\begin{frame} 

	\begin{block}{Multimedia} 
	\begin{itemize}
	\item The GIMP
	\newline Bildbearbeitung (Photoshop)
	\item Inkscape
	\newline Vektorbasiertes Grafikprogramm (Illustrator)
	\item ImageMagick
	\newline Kommandlinetool um Bilder zu verändern (\textit{unvergleichlich})
	\item Audacity
	\newline Audioeditor (Soundbooth)
	\item Cinelerra / KDenlive / OpenShot
	\newline Videoeditor (Premiere)
	\item Blender
	\newline 3d Animationstool (Maya) 
	\end{itemize}

	\end{block}
	
\end{frame}

\begin{frame} 

	\begin{block}{Server} 
	\begin{itemize}
	\item Apache / lighttpd / ...
	\newline Ausliefern von Webseiten
	\item MySQL / PostgreSQL / ...
	\newline Datenbank Server
	\item Postfix
	\newline E-Mail Server
	\item Samba / NFS
	\newline Windows / Linux Datenfreigabe
	\item OpenLDAP / Kerberos
	\newline Verzeichnissdienst, Authentifizierung
	\item DNS (Bind) / DHCP
	\newline Namesauflösung / Versorgung mit IP-Adressen
	\end{itemize}

	\end{block}
	
\end{frame}

\begin{frame} 

	\begin{block}{Programmieren und Entwickeln} 
	\begin{itemize}
	\item Allein durch die GNU-Compiler sind viele Programmiersprachen möglich
	\newline C, C++, Objective C, Java, Fortran, Ada
	\item So gut wie alle Skriptsprachen sind verfügbar:
	\newline Python, Perl, Ruby, Lua, PHP, ...
	\item Verschieden Entwicklungsstile möglich
	\newline Von Eclipse bis vi
	\item Natürlich sind auch Versionsverwaltungsprogramme vorhanden
	\newline SVN, Git, Mercurial, ...
	\end{itemize}

	\end{block}
	
\end{frame}

%%%%%%%%%%%%%%%%%%%%%%%%%%

\begin{frame}[plain]
\begin{alertblock}{Nächste Vorlesung}
\textbf{Termin:} In 15 Minuten\\
\textbf{Thema:} Kapitel 2 - Erste Schritte mit Linux \\
\textbf{Lehrbuchkapitel:} 
\begin{itemize}
\item LXES 1 Computer, Software und Betriebssysteme
\item LXES 2 Linux und freie Software
\item LXES 3 Erste Schritte mit Linux
\end{itemize}
\end{alertblock}
\end{frame}

\materialframe

\end{document}
