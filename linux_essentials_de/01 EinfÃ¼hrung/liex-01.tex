\documentclass[aspectratio=43]{beamer}
\usepackage{../../resources/agdsncourse} 

\title[Linux Essentials  - Kapitel 1 - Einführung]{Linux Essentials}
\subtitle{Kapitel 1 - Einführung}
\author{Hagen Eckert}
\date{17. Oktober 2015}


\begin{document}

% Start Folie
\logoframe

%Title
\frame{\titlepage}

%Gliederung
\setcounter{tocdepth}{1}
\section[Gliederung]{}
\frame{\tableofcontents}


%%%%%%%%%%%%%%%
\section{Organisatorisches}
\subsection{Ziele \& Vorlesende}
\begin{frame} 

	\begin{block}{Ziele} 
	\begin{itemize}
	\item Grundwissen zur kompetenten Benutzung von Linux
	\item Einblicke in freie Software
	\item Aufbau und Zusammenspiel der Komponenten von Linux
	\item Grundlagen zur Benutzung der Kommandozeile
	\item Einführung in die Systemadministration
	\end{itemize}
	\end{block}
	
	\begin{block}{Vorlesende} 
	\begin{itemize}
	\item Alexander Köhler
	\item Hagen Eckert
	\end{itemize}
	\end{block}

\end{frame}

\subsection{LPI}
\begin{frame} 

	\begin{block}{LPI} 
	\begin{itemize}
	\item LPI $\rightarrow$ \textbf{L}inux \textbf{P}rofessional \textbf{I}nstitute
	\item Weltweit größtes Zertifizierungsprogramm der Linux-Gemeinschaft
	\item Distributionsunabhängig
	\end{itemize}
	\end{block}
	
	\begin{block}{LPI AAP} 
	\begin{itemize}
	\item Die AG DSN ist \textbf{A}pproved \textbf{A}cadamic \textbf{P}artner
	\item Bald auch privates Prüfungscenter
	\end{itemize}
	\end{block}
\end{frame}

\subsection{Vorlesung}
\begin{frame} 
	\begin{block}{Vorlesung} 
	\begin{itemize}
	\item immer Donnerstag
	\item Uhrzeit entweder 18:30 - 20:00 Uhr (jetzt)
	\newline oder 16:40 - 18:10 Uhr
	\item 10x Vorlesung
	\item Wiederholung + Prüfung
	\end{itemize}
	\end{block}
	
	\begin{block}{Prüfung} 
	\begin{itemize}
	\item Computer basierte Prüfung 
	\item Sofortige Rückmeldung zum Ergebnis 
	\item Kosten für die Zertifizierung sind 50 EUR
	\item Aktive AG DSN Mitglieder werden mit 30 EUR subventioniert
	\end{itemize}
	\end{block}
\end{frame}

\subsection{Lehrstuhl}

\begin{frame}
	\begin{block}{Lehrstuhl} 
	\begin{itemize}
	\item Materialwissenschaft und Nanotechnik  
	\item Prof. Cuniberti
	\item Fokus: Konzepte, quantitative Methoden und Experimente um Phänomene im Nanometerbereich zu verstehen und zu beherrschen
	\item Für die Lösung vieler Fragen werden Hochleistungsrechner eingesetzt  $\rightarrow$ 94\% der Hochleistungsrechner laufen unter Linux
	\end{itemize}
	\end{block}
	
\end{frame}

\begin{frame}
	\begin{block}{Lehrstuhl} 
	\begin{itemize}
	\item Institut für Strahlenphysik, Abteilung~Laser-Teilchenbeschleunigung, Helmholtz-Zentrum Dresden - Rossendorf  
	\item Prof. Schramm
	\item Mehrere Gruppen:
	\begin{itemize}
	 \item Laser-Elektronenbeschleunigung und Thomsonstreuung
	 \item Computergestützte Strahlenphysik \\ $\rightarrow$ GPU-Programmierung PIConGPU zu 100\% unter Linux
	 \item \dots
	\end{itemize}
% 	\item Hochleistungsrechner eingesetzt  $\rightarrow$  Hochleistungsrechner laufen unter Linux
	\end{itemize}
	\end{block}
	
\end{frame}

%%%%%%%%%%%%%%%%%
\section{Der Computer}
\subsection{Geschichte}
\begin{frame} 

	\begin{block}{Geschichte} 
	\begin{itemize}
	\item Beginn schwer abzugrenzen
	\item Antike Rechenhilfen und mathematische Methoden bilden dir Grundlage der Entwicklung hin zum modernen Computer
	\item Im 17. Jahrhundert kommen erste mechanische Rechner auf
	\item Das binäre Zahlensystem wird 1703 von Leibniz entwickelt
	\item Lochkarten erlauben das \textit{programmieren}
	\newline (1890 US-Volkszählung)
	\item 1941 ist die Zuse Z3 die erste Maschine die beliebige Algorithmen automatisch ausführen konnte.
	\end{itemize}
	\end{block}


\end{frame}

\subsection{Hardware}
\begin{frame} 

	\begin{block}{Prozessor} 
	\begin{itemize}
	\item \textbf{C}entral \textbf{P}rocessing \textbf{U}nit
	\item Besteht aus:
	\begin{itemize}
	 \item Arimethrisch-logische Einheit (ALU)
	 \item Steuerwerk
	 \item Register
	 \item Speichermanager (MMU)
	
	\end{itemize}
	\item Verschiedene Architekturen wie X86-64, ARM
	\end{itemize}
	\end{block}

	\begin{block}{RAM} 
	\begin{itemize}
	\item \textbf{R}andom-\textbf{A}ccess \textbf{M}emory
	\item Programmcode und zu verarbeitende Daten werden hier gespeichert
	\end{itemize}
	\end{block}
\end{frame}

\begin{frame} 

	\begin{block}{Coprozessor} 
	\begin{itemize}
	\item Sollen die CPU von verschiedenen Aufgaben entlasten
	\item Effizienter durch Spezialisierung 
	\item Grafikkarten, Soundkarten, IO-Erweiterungen, ...
	\item auch allgemein gehaltene Prozessoren möglich
	\end{itemize}
	\end{block}

	\begin{block}{Mainboard} 
	\begin{itemize}
	\item Bildet die Infrastruktur für die restliche Hardware
	\item Stellt Datenverbindungen zwischen den Komponenten her
	\item Verteilt Strom
	\item Besitzt eigenen Chips insbesondere für IO-Aufgaben
	\end{itemize}
	\end{block}
\end{frame}

\begin{frame} 

	\begin{block}{Speicher} 
	\begin{itemize}
	\item Dauerhafter Datenspeicher
	\item Für hohe Kapazität immer noch Festplatten am wichtigsten
	\item Durch die mechanischen Teile jedoch einige Nachteile
	\item Solid State Disk werden immer wichtiger, insbesondere wenn schnelle Obertragungs - und Zugriffsgeschwindigkeiten nötig sind
	\item Denoch um Größenordungen langsamer als der RAM 
	\begin{itemize}
	 \item[RAM] 6$\cdot10^{-8}$ s
	\item[SSD] 2,5$\cdot10^{-4}$ s
	\item[HD] $10^{-2}$ s
	\end{itemize}

	
	\end{itemize}
	\end{block}

\end{frame}


\begin{frame} 

	\begin{block}{Optische Laufwerke} 
	\begin{itemize}
	\item CD, DVD, Blu-ray
	\item Daten werden mit Laser gelesen $\rightarrow$ optisch
	\item verlieren stark an Bedeutung durch USB-Sticks und Breitband Internet
	\end{itemize}
	\end{block}

	\begin{block}{Netzwerk} 
	\begin{itemize}
	\item Verschiedene Schnittstellen möglich
	\item Von 10BASE BNC bis 10Gbit/s Glasfaser möglich
	\item Meistens ist Gigabit-Ethernet über Kupfer oder verschiedene WLAN-Varianten anzutreffen
	\item Glasfaser meist nur für Backbones und Server verwendet
	\end{itemize}
	\end{block}
\end{frame}

\subsection{Betriebssysteme}


\begin{frame} 

	\begin{block}{Windows} 
	\begin{itemize}
	\item Marktanteil um 82 \% (Internet)
	\item 1985 als grafischer Aufsatz für MS-DOS gestartet
	\item Seit Windows NT eigenständiges Betriebssysteme
	\end{itemize}
	\end{block}
	
	\begin{block}{OS X} 
	\begin{itemize}
	\item Von Grundsystem näher an Linux
	\item Vor allem durch GNU Programme
	\item Abstammung ebenfalls von UNIX durch  BSD 
	\end{itemize}
	\end{block}

\end{frame}

\begin{frame} 

	\begin{block}{Linux} 
	\begin{itemize}
	\item Linux stellt genau genommen nur ein Betriebssystems Kern dar
	\item Es wird dennoch als Sammelbegriff für das gesamte System genutzt
	\newline Nur selten wird GNU/Linux im allgemeinen Sprachgebrauch genutzt
	\item Entwicklung geht auf Linus Torvalds zurück
	\item 1991 als Hobby Projekt begonnen
	\item Nachbildung des Minix Systems
	\item Kombiniert mit den GNU Programmen entstand ein komplettes Betriebssystem 
	\end{itemize}
	\end{block}

\end{frame}

\begin{frame} 

	\begin{block}{Linux} 
	\begin{itemize}
	\item Linux 2.0 war sehr wichtig da Mehrkernprozessoren und das dynamische Laden von Modulen erlaubt
	\item Ständige Weiterentwicklung bis heute
	\item Portiert auf fast alles was \textit{rechnen} kann
	\item Im allgemeinem PC Markt nie besonders erfolgreich
	\newline Erfolge eher im Bereich Embedded Systeme, Mobile Anwendungen, Server, Wissenschaftliche Anwendung
	\end{itemize}
	\end{block}

\end{frame}

\section{Software}
\subsection{Distributionen}
\begin{frame} 

	\begin{block}{Distributionen} 
	\begin{itemize}
	\item Stellen komplette Softwareumgebungen bereit
	\item Kommandozeilenwerkzeuge, Arbeitsumgebung, Officeprogramme, Spiele, ...
	\item Viele verschiedene, angepasst an unterschiedliche Bedürfnisse und Philosophien 
	\end{itemize}
	\end{block}

	\begin{block}{Red Hat} 
	\begin{itemize}
	\item Gegründet 1993, aufkauf der Red Hat Distribution 1995
	\item Wohl größte Firma auf Basis von Linux und Open Source
	\item Verdient ihr Geld mit Support und speziellen Updateservices
	\item Eigene \textit{Testumgebung} Fedora
	\item Nachbauten wie CentOS
	\end{itemize}
	\end{block}

	
\end{frame}

\begin{frame} 

	\begin{block}{SUSE} 
	\begin{itemize}
	\item Basiert auf der ersten Linux-Disto SLS
	\item Bedeutend im deutschsprachigen Raum
	\item Wie Red Hat spezielle Enterprise Edition
	\item openSUSE für Endkunden
	\item Besonderheit ist YaST
	\end{itemize}
	\end{block}

	\begin{block}{Debian} 
	\begin{itemize}
	\item Ohne Firma im Hintergrund
	\item Begonnen 1993
	\item Strenge Regeln was freie Software ist
	\item Dient häufig als Grundlage anderer Projekte
	\end{itemize}
	\end{block}

	
\end{frame}

\begin{frame} 

	\begin{block}{Ubuntu} 
	\begin{itemize}
	\item Basiert auf Debian unstable
	\item Kürzere Erscheinungszyklen als Debian
	\item Mit den LTS-Versionen soll Konkurrenz zu etablierten Server Systemen geschaffen werden
	\item Unterstützt durch Canonical Ltd.
	\end{itemize}
	\end{block}

	\begin{block}{Weiter Distributionen} 
	\begin{itemize}
	\item Gentoo Linux $\rightarrow$ Quellcode
	\item Archlinux
	\item \textit{Android}
	\end{itemize}
	\end{block}

	
\end{frame}
\subsection{Anwendungsprogramme}
\begin{frame} 

	\begin{block}{Büroprogramme} 
	\begin{itemize}
	\item OpenOffice.org 
	\begin{itemize}
	 \item Schon viele Jahre bedeutendste freie Büropacket
	 \item Über SUN und Oracle bei der Apache Software Foundation gelandet
	\end{itemize}
	\item LibreOffice
	\begin{itemize}
	 \item Abspaltung von OpenOffice
	 \item Gilt als Innovativer und Aufgeräumter
	\end{itemize}


	\end{itemize}
	\end{block}

	\begin{block}{Internet} 
	\begin{itemize}
	\item Firefox
	\item Thunderbird
	\item Chromium
	\end{itemize}
	\end{block}

	
\end{frame}

\begin{frame} 

	\begin{block}{Multimedia} 
	\begin{itemize}
	\item The GIMP
	\newline Bildbearbeitung (Photoshop)
	\item Inkscape
	\newline Vektorbasiertes Grafikprogramm (Illustrator)
	\item ImageMagick
	\newline Kommandlinetool um Bilder zu verändern (\textit{unvergleichlich})
	\item Audacity
	\newline Audioeditor (Soundbooth)
	\item Cinelerra / KDenlive / OpenShot
	\newline Videoeditor (Premiere)
	\item Blender
	\newline 3d Animationstool (Maya) 
	\end{itemize}

	\end{block}
	
\end{frame}

\begin{frame} 

	\begin{block}{Server} 
	\begin{itemize}
	\item Apache / lighttpd / ...
	\newline Ausliefern von Webseiten
	\item MySQL / PostgreSQL / ...
	\newline Datenbank Server
	\item Postfix
	\newline E-Mail Server
	\item Samba / NFS
	\newline Windows / Linux Datenfreigabe
	\item OpenLDAP / Kerberos
	\newline Verzeichnissdienst, Authentifizierung
	\item DNS (Bind) / DHCP
	\newline Namesauflösung / Versorgung mit IP-Adressen
	\end{itemize}

	\end{block}
	
\end{frame}

\begin{frame} 

	\begin{block}{Programmieren und Entwickeln} 
	\begin{itemize}
	\item Allein durch die GNU-Compiler sind viele Programmiersprachen möglich
	\newline C, C++, Objective C, JAve, Fortran, Ada
	\item Sogut wie alle Skriptsprachen sind verfügbar:
	\newline Python, Perl, Ruby, Lua, PHP, ...
	\item Verschieden Entwicklungsstiele möglich
	\newline Von Eclipse bis vi
	\item Natürlich sind auch Versionsverwaltungsprogramme vorhanden
	\newline SVN, Git, Mercurial, ...
	\end{itemize}

	\end{block}
	
\end{frame}

%%%%%%%%%%%%%%%%%%%%%%%%%%

\begin{frame}[plain]
\begin{alertblock}{Nächste Vorlesung}
\textbf{Termin:} Donnerstag, den 18.04.2013 18:30-20:00 Uhr\\
\textbf{Thema:} Kapitel 2 - Erste Schritte mit Linux \\
\textbf{Vorbereitung:} 
\begin{itemize}
\item LXES 1 Computer, Software und Betriebssysteme
\item LXES 2 Linux und freie Software
\item LXES 3 Erste Schritte mit Linux
\end{itemize}
\end{alertblock}
\end{frame}

\materialframe

\end{document}
