\documentclass[aspectratio=43]{beamer}
\usepackage{../../resources/agdsncourse} 

\title[Linux Essentials  - Kapitel 10 - Linux im Netz]{Linux Essentials}
\subtitle{Kapitel 10 - Linux im Netz}
\author{Alexander Köhler}
\date{25. Oktober 2015}
\setbeamertemplate{navigation symbols}{} 
\begin{document}

%%%%%%%%%%%%%%%%%%%%%%%5
% Start Folie
\logoframe

%Title
\frame{\titlepage}



%Gliederung
\setcounter{tocdepth}{1}
\section[Gliederung]{}
\frame{\tableofcontents}


%%%%%%%%%%%%%%% der Inhalt
\section{Benutzerverwaltung -- Teil II}
\subsection{Dateien zur Gruppenverwaltung}
\begin{frame}{\co{/etc/shadow}}
  \begin{center}
    \colorbox{yellow}{\co{Name:Kennwort:Änderung:Min:Max:Warn:Frist:Sperre:Reserve}}
  \end{center}
  \begin{block}{}
    \begin{description}
      \item[Name]        Textueller Nutzername, Login
      \item[Kennwort]    \dots
      \item[Änderung]    Datum der letzten Passwortänderung
      \item[Min]         Mindestalter in Tagen
      \item[Max]         Höchstalter in Tagen
      \item[Warn]        Tage vor Max für Aufforderung zum Ändern 
      \item[Frist]       Tage nach Max, Sperrung bei Überschreitung
      \item[Sperre]      Datum für Ablauf des Accounts
    \end{description}
  \end{block}
\end{frame}
\begin{frame}{\co{/etc/group}}
  \begin{center}
    \colorbox{yellow}{\co{Gruppenname:Kennwort:GID:Mitglieder}}
  \end{center}
  \begin{block}{}
    \begin{description}
      \item[Gruppenname] Textueller Name der Gruppe
      \item[Kennwort]    Wird beim Wechseln der Gruppe mit \co{newgrp} abgefragt, \co{*} und \co{x} wie bei \co{passwd} 
      \item[GID]         Numerische Gruppenkennung
      \item[Mitglieder]  Durch Komma getrennte Liste der Mitglieder (Sekundäre Gruppenmitglieder)
    \end{description}
  \end{block}
    Kennwort und Mitglieder sind optional
\end{frame}
\begin{frame}{\co{/etc/gshadow}}
%   \co{}
  \begin{block}{Gruppenpasswörter}
    \begin{itemize}
      \item werden im Allgemeinen nicht verwendet
      \item \co{/etc/gshadow}
    \end{itemize}
  \end{block}
\end{frame}

\subsection{Benutzerkonten erstellen}
\begin{frame}{Überblick: Neuen Nutzer anlegen}
%   $\rightarrow$ Aufgabe als root

  \begin{block}{Aufgaben beim Anlegen neuer Benutzer}
    \begin{enumerate}
      \item Einträge in \co{/etc/\{passwd,shadow\}} anlegen
      \item Ggf. Eintrag in \co{/etc/group} anlegen
      \item Heimatverzeichnis erzeugen und Grundausstattung mitgeben
      \item Ggf. weitere Dateien ändern (z.B. Plattenkontigente, Zugriffsberechtigungen)
    \end{enumerate}
  \end{block}
  $\rightarrow$ Bearbeiten der Textdateien von Hand möglich, aber z.B. Tippfehler möglich\\
  $\rightarrow$ Elegantere Methode: Programm \co{useradd}
\end{frame}

\begin{frame}{Einen neuen Nutzer anlegen}
  \begin{center}
    \colorbox{yellow}{\co{\# useradd [Optionen] Nutzername}}
  \end{center}
  \begin{block}{Optionen}
    \begin{description}
      \item[-c Kommentar] Eintrag im GECOS-Feld
      \item[-d Heimatverzeichnis] sonst wird \co{/home/Nutzername} verwendet
      \item[-e Datum] Sperrdatum, Format: ``JJJJ-MM-DD''
      \item[-g Gruppe] Primäre Gruppe
      \item[-G Gruppe,\dots] Sekundäre Gruppen
      \item[-s Shell] Login-Shell
      \item[-u UID] Numerische Nutzerkennung
      \item[-m] Heimatverzeichnis mit Grundausstattung erstellen
      \item[-k Skel-Verzeichnis] Angabe des Skeleton-Verzeichnisses (\co{/etc/skel})
    \end{description}
  \end{block}
\end{frame}
\begin{frame}{Hinweise zu \co{useradd}}
  \begin{block}{Voreingestellte Werte anzeigen und ändern}
    \begin{itemize}
      \item Anzeigen: \co{\#~useradd~-D} {\color{gray} (bei openSUSE \co{--show-defaults})}
      \item Ändern: durch Anhängen der Optionen
    \end{itemize}
  \end{block}
  \begin{block}{Weitere Hinweise}
    \begin{itemize}
      \item relativ niedriger Administrationslevel \\
            $\rightarrow$ ggf. eigenes Shellskript für umfangreichere Konfiguration
      \item Distributionseigene Varianten!
      \item Neues Konto noch ohne gültiges Passwort / Login
      \item Ohne Option ``-m'' auch kein existierentes Heimatverzeichnis
    \end{itemize}
  \end{block}
\end{frame}
\begin{frame}{Passwörter mit \co{passwd} einrichten}
  \begin{center}
    \colorbox{yellow}{\co{\$~passwd~[Optionen]~[Nutzername]}}
  \end{center}
  \begin{block}{Ohne Optionen}
    \begin{itemize}
      \item Änderung des Passwort des (angemeldeten) Nutzers
      \item Zuvor Abfrage des alten Passworts (außer bei root)
    \end{itemize}
  \end{block}
  \begin{block}{Optionen}
    \begin{description}
      \item[-S] Passwort-Verwaltungsdaten anzeigen
      \item[-l] Sperren
      \item[-u] Entsperren
      \item[\dots]
    \end{description}
  \end{block}
\end{frame}
\begin{frame}{Feinere Passwort-Einstellungen mit \co{chage}}
  \begin{center}
    \colorbox{yellow}{\co{\#~chage~[Optionen]~Nutzername}}
  \end{center}
  \begin{block}{Hinweise zu den Optionen}
    \begin{itemize}
      \item Ohne Optionen: interaktive Abfrage aller Werte
      \item Alle Einstellungen in \co{/etc/shadow} änderbar, nicht nur die durch \co{passwd} verändert werden können
      {\color{gray} \item Optionen nicht mit denen von \co{passwd} kompatibel}
    \end{itemize}
  \end{block}
\end{frame}

\begin{frame}{Passwort vergessen -- Was tun?}
    \begin{itemize}
      \item Passwort immer nur verschlüsselt gespeichert \\
      $\rightarrow$ Direktes Auslesen des Passworts in Klartext technisch unmöglich
      \item Passwort durch root mit \co{passwd} neu setzen
      \item root-Passwort durch Booten mit einer Live-CD ändern / löschen
    \end{itemize}
\end{frame}

\subsection{Benutzerkonten löschen}
\begin{frame}{Benutzerkonten löschen}
  \begin{block}{Konto löschen}
    \begin{itemize}
      \item \co{\#~userdel~Nutzer} \\
      $\rightarrow$ entfernt Einträge in \co{/etc/\{passwd,shadow\}}
      \item Option \co{-r} entfernt auch Heimatverzeichnis und Mailbox (\co{/var/mail})
    \end{itemize}
  \end{block}
  \begin{alertblock}{Weitere Hinweise}
    Dateien an anderen Orten müssen explizit gelöscht werden\\
    $\rightarrow$ \co{find~/~-uid~UID~-delete}
  \end{alertblock}
\end{frame}

\begin{frame}{Benutzerkonten ändern}
  \begin{block}{Konto ändern mit \co{usermod}}
    \begin{center}
      \colorbox{yellow}{\co{\$~usermod~Optionen~Nutzername}}
    \end{center}
    $\rightarrow$ Optionen wie bei \co{useradd}
  \end{block}
  \begin{block}{UID verändern}
    \begin{itemize}
      \item Möglich durch direkte Änderung des Feldes in \co{passwd}
      \item Dateien werden nicht automatisch übertragen!\\
        $\rightarrow$ Übertragen der Eigentümerschaft mit \\
        \co{\# chown~-R~Verzeichnis}
    \end{itemize}
  \end{block}
\end{frame}

\begin{frame}{Hinzufügen, Bearbeiten, Löschen}
  \begin{block}{Gruppe hinzufügen}
    \co{\#~groupadd~Gruppenname}\\
    $\rightarrow$ Option ``-g GID'' legt GID fest 
%     (System: 0\dots99, Normale Gruppen: 100\dots)
  \end{block}
  \begin{block}{Gruppe bearbeiten}
    \co{\#~groupmod~[-g~GID][-n~Name]~Gruppenname} 
  \end{block}
  \begin{block}{Gruppe löschen}
    \co{\#~groupdel~Gruppenname}\\
    $\rightarrow$ Dateien der Gruppe werden zurückgelassen\\
    $\rightarrow$ \co{chgr}
  \end{block}
\end{frame}
\begin{frame}{Einstellungen zum Gruppenzugang ändern}
  \begin{block}{\co{gpasswd}}
    \co{\#~gpasswd~[Optionen]~Gruppenname}\\
    $\rightarrow$ Ändert Gruppenpasswort
    \begin{description}
      \item[-a~Nutzer] Nutzer zu Gruppe hinzufügen
      \item[-d~Nutzer] Nutzer aus Gruppe entfernen
      \item[-A~Nutzer] Gruppenadministrator hinzufügen
    \end{description}
  \end{block}
\end{frame}

\begin{frame}{Einstellungen direkt über Texteditor ändern}
  \begin{itemize}
    \item Bearbeitung durch direkten Aufruf eines Texteditors möglich\\
    $\rightarrow$ Problem: Datei nicht gesperrt und Änderungen können überschrieben werden
    \item \co{vipw} bzw. \co{vigr} öffnen und sperren \co{/etc/\{passwd,group\}} \\
    $\rightarrow$ Geschützt vor einem Überschreiben
    \item Mit ``-s'' werden die entsprechenden Schattenpasswort-Dateien genutzt
    \item Editor meistens \co{vi}, einstellbar über Umgebungsvariablen \co{VISUAL} bzw. \co{EDITOR}
  \end{itemize}
\end{frame}


\section{Linux im Netz -- Teil II}
\subsection{Linux als Client im Netzwerk}
\begin{frame}{}
  \begin{alertblock}{Nötige Informationen zum Netzanschluss}
    \begin{itemize}
      \item IP-Adresse (für den Rechner)
      \item Netzwerkadresse und -maske
      \item IP-Adresse des Default-Gateways
      \item ein oder zwei IP-Adressen von DNS-Servern
    \end{itemize}
  \end{alertblock}
  \begin{exampleblock}{Idealfall}
    Automatische Konfiguration über DHCP oder ggf. beim Starten des Systems durch Startskripte\\[.7em]
    {\color{gray}(siehe \co{/etc/network/interfaces/} und \co{/etc/sysconfig/network\{,-scripts\}/})}
  \end{exampleblock}
\end{frame}

\begin{frame}{Temporäre Konfigurationsmöglichkeiten}
  \hspace*{-2em}
  \begin{block}{Netz}
    \co{\#~ifconfig~eth0~192.168.178.11~netmask~255.255.255.128}\\
  \end{block}
  \begin{block}{Route}
    \co{\#~route~add~-net~default~gw~192.168.178.1}\\
  \end{block}
  \begin{block}{DNS Einstellungen in \co{/etc/resolv.conf}}
    \begin{description}
      \item[search~server.com ] Std.-Erweiterung bei Namen ohne ``.''
      \item[nameserver~IP-Addr] DNS-Server Adresse (1\dots 3)
    \end{description}
    $\rightarrow$ Datei kann von anderen Programmen überschrieben werden!
  \end{block}
\end{frame}

\subsection{Fehlersuche}

\begin{frame}{Mit \co{ping} die Erreichbarkeit prüfen}
  \begin{block}{Grundlagen}
    \begin{itemize}
      \item Versendet ICMP-Pakete an Ziel-IP
      \item Voraussetzung: Quell- und Zielrechner können Pakete voneinander empfangen
      \item Mgl. Problem: ICMP-Pakete von Firewall geblockt
    \end{itemize}
  \end{block}
  \begin{block}{Überprüfen der Erreichbarkeiten}
    \begin{enumerate}
      \item Loopback-Schnittstelle (127.0.0.1)
      \item Lokale Schnittstelle
      \item Default-Gateway
      \item Jenseits des Gateways (Internet)
    \end{enumerate}
  \end{block}
\end{frame}

\begin{frame}{Namensauflösung kontrollieren mit \co{dig}}
  \begin{block}{Testen der Namensauflösung}
    \co{\$~dig~[@DNSserver]~name~[Typ]}
    \begin{itemize}
      \item \co{dig~www.tu-dresden.de}
      \item \co{dig~-x~8.8.8.8}
      \item Option ``+trace'': Gesamten Suchkette ausgeben
    \end{itemize}
  \end{block}
\end{frame}

\begin{frame}{Mit \co{netstat} mehr über die Netzanbindung herausfinden}
  \begin{block}{Allgemeines zu \co{netstat}}
    \begin{itemize}
      \item Ohne Optionen: Listet alle Verbindungen auf (Unix-Domain-Sockets, TCP und UDP)
      \item Nur offene TCP-Verbindungen: \co{netstat~-tl}
%           $\rightarrow$ alle 
      \item Liste der Portnummern: \co{/etc/services}
    \end{itemize}
  \end{block}
  \begin{block}{Weitere Anzeigeoptionen}
    \begin{itemize}
      \item Statistikinformationen: ``-s''
      \item Routen: ``-r''
      \item Netzwerkschnittstellen: ``-i''
      \item Aktualisierte Anzeige: ``-c''
    \end{itemize}
  \end{block}
\end{frame}

\begin{frame}{Prüfungsziele}
  \begin{alertblock}{5.1 Grundlegende Sicherheit und Identifizierung von Benutzertypen}
    \begin{description}
      \item[Gewicht]  2
      \item[Beschreibung] Verschiedene Benutzertypen in einem Linux-System
    \end{description}
       Wissensgebiete\\ 
        \begin{itemize}
          \item Root, unprivilegierte Nutzer, Systemnutzer
          \item \co{/etc/passwd}, \co{/etc/group}
          \item \co{id}, \co{who}, \co{w}
          \item \co{sudo}
        \end{itemize}
  \end{alertblock}
\end{frame}
\begin{frame}{Prüfungsziele}
  \begin{alertblock}{5.2 Benutzer und Gruppen erstellen}
    \begin{description}
      \item[Gewicht]  2
      \item[Beschreibung] Benutzer und Gruppen in einem Linuxsystem erstellen
    \end{description}
       Wissensgebiete\\ 
        \begin{itemize}
          \item Befehle zu Nutzern und Gruppen
          \item UIDs
          \item \co{/etc/passwd}, \co{/etc/shadow}, \co{/etc/group}
          \item \co{id}, \co{last}
          \item \co{useradd}, \co{groupadd}, \co{passwd}
        \end{itemize}
  \end{alertblock}
\end{frame}
\begin{frame}{Prüfungsziele}
  \begin{alertblock}{4.4 Den Computer ins Netz einbinden}
    \begin{description}
      \item[Gewicht]  2
      \item[Beschreibung] Abfragen wesentlicher Netzwerkeinstellungen und Bestimmen der 
              Grundvoraussetzungen für einen Computer in einem LAN
    \end{description}
       Wissensgebiete\\ 
        \begin{itemize}
          \item Netzwerkeinstellungen, DNS, Internet
          \item IPv4, IPv6
          \item \co{ifconfig}, \co{route}, \co{resolv.conv}
          \item \co{netstat}, \co{dig}, \co{ping}
        \end{itemize}
  \end{alertblock}
\end{frame}
%%%%%%%%%%%%%%%%%%%%%%%%%% next time...
\begin{frame}[plain]
  \begin{alertblock}{Nächste Vorlesung}
    \textbf{Termin:} Donnerstag, den 04.07.2013 18:30-20:00 Uhr\\
    \textbf{Thema:} Kapitel 11 - Wiederholung\\
    \textbf{Vorbereitung:} 
    \begin{itemize}
%       \item 
      \item Gedanken und Fragen zur Vorlesung machen
    \end{itemize}
  \end{alertblock}
\end{frame}



\materialframe

\end{document}
