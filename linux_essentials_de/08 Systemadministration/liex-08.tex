\documentclass[aspectratio=43]{beamer}
\usepackage{../../resources/agdsncourse} 

\title[Linux Essentials  - Kapitel 8 - Systemadministration]{Linux Essentials}
\subtitle{Kapitel 8 - Systemadministration}
\author{Alexander Köhler}
\date{24. Oktober 2015}
\begin{document}

%%%%%%%%%%%%%%%%%%%%%%%
% Start Folie
\logoframe

%Title
\frame{\titlepage}

%Gliederung
\setcounter{tocdepth}{1}
\section[Gliederung]{}
\frame{\tableofcontents}


%%%%%%%%%%%%%%% der Inhalt
\section{Einstieg in die Systemadministration}
\subsection{Grundlagen}
\begin{frame}{}
  \begin{block}{Der \co{root}-Nutzer}
    \begin{itemize}
      \item Administratorkonto unter Linux
      \item von Rechteprüfungen ausgenommen
      $\rightarrow$ \textbf{darf alles}
      \item Mit großer Macht kommt große Verantwortung!
    \end{itemize}
  \end{block}

  \begin{block}{Arbeiten als \co{root}}
    \begin{itemize}
      \item Menschenverstand nutzen
      \item Graphische Anmeldung möglichst vermeiden
      \item Keine normalen Programme unnötig nutzen (Mail, Browser)
    \end{itemize}
  \end{block}
    \begin{alertblock}{Zu \co{root} werden}
    \begin{itemize}
      \item \co{su~-} wechselt nach Kennwortaufforderung in  \co{root}-Shell
      \item \co{sudo befehl} führt ``\co{befehl}'' mit \co{root}-Rechten aus
    \end{itemize}
  \end{alertblock}
\end{frame}

\subsection{Systemkonfiguration}
% \begin{frame}{}
%   \begin{block}{}
%     \begin{itemize}
%       \item 
%     \end{itemize}
%   \end{block}
% \end{frame}

\begin{frame}{Konfiguration in Textdateien}
  \begin{block}{}
    Systemweite Konfiguration findet in versch. Dateien in \co{/etc} statt
    \begin{itemize}
      \small
      \item Benutzerdaten in \co{/etc/passwd} ändern
      \item Festplatten einbinden in \co{/etc/fstab}
    \end{itemize}
  \end{block}
  \begin{block}{Vorteile}
    \begin{itemize}
      \item Fehler in einer Datei legen nicht das ganze System lahm
      \item Kommentare (\co{\#}) erlaubt $\rightarrow$ Dokumentation von Änderungen
      \item Verwaltung durch Revisionskontrollsystem möglich
      \item bequeme Konfiguration ganzer Rechnernetze z.B. durch Schablonen
    \end{itemize}
  \end{block}
\end{frame}


\subsection{Prozesse unter Linux}
\begin{frame}{Prozesse}
  \begin{block}{Was ist ein Prozess?}
    \begin{itemize}
      \item Programm, dass vom Betriebssystem ausgeführt wird
      \item besitzt Daten im Arbeitsspeicher, Verwaltungsinformationen, Berechtigungen,
            Prozessumgebung (Umgebungsvariablen), \textbf{eine eindeutige Prozessnummer (PID)}
    \end{itemize}
  \end{block}
  \begin{exampleblock}{Der erste Prozess}
    \dots ist \co{/sbin/init} mit der PID 1
  \end{exampleblock}
\end{frame}
\begin{frame}{Prozesse mit \co{ps} auflisten}
  \begin{block}{Befehl \co{ps}}
    \begin{itemize}
      \item listet Prozesse des Nutzers auf bzw. den mit der PID
      \item mehr Informationen mit z.B. \co{ps l}
    \end{itemize}
  \end{block}
  \begin{block}{Prozesszustand}
    \begin{description}
      \item[Lauffähig (R):]     läuft normal
      \item[Schlafend (S):]     warten auf Eingabe, Ausgabe, \dots
      \item[Im Tiefschlaf (D):] wartet auf Ereignis und kann nicht unterbrochen werden
      \item[Gestoppt (T):]      vom Eigentümer angehalten
      \item[Zombie (Z):]        Rückgabewerte müssen noch gelesen werden
    \end{description}
  \end{block}
\end{frame}

\begin{frame}{Speicherauslastung mit \co{free} anzeigen}
  \begin{block}{Befehl \co{free}}
    \begin{itemize}
      \item Ausgabe der aktuellen Speicherbelegung
      \item \co{Mem} entspricht RAM
      \item \co{-/+} entspricht RAM korrigiert um Puffer und Cache
      \item \co{Swap} entspricht der Auslagerungspartition \\
            $\rightarrow$ Linux nutzt eine seperate Partition dafür
            (vgl.~Auslagerungsdatei~oder~Pagefile)
    \end{itemize}
  \end{block}
  \begin{exampleblock}{Beispiel}
      \co{\$~free \\}
  \end{exampleblock}
\end{frame}

\begin{frame}{top}
  \begin{block}{Befehl \co{top}}
    \begin{itemize}
      \item Interaktive und kontinuierliche Auflistung der vorhandenen Prozesse
      \item Auslastung der einzelnen CPU-Kerne
      \item Signale an Prozesse senden (z.B. Beenden, Stoppen, ...)
      \item Laufzeitparameter ändern (Nice-Level / Priorität)
    \end{itemize}
  \end{block}
\end{frame}

\begin{frame}{Das \co{/proc}-Verzeichnis}
  \begin{block}{Dateien}
    \begin{description}
      \item[``ZAHL''] Verzeichnis mit Daten zum Prozess mit PID ``ZAHL''
      \item[cpuinfo] Typ und Taktfrequenz der CPU
      \item[devices] aktuell unterstützte Geräte
      \item[interrupts] Hardwareinterrupts
      \item[kcore]   Direktzugriff auf den Arbeitsspeicher!!!
  %     \item[dma]     
      \item[loadavg] Auslastung der CPU in den letzten 1, 5 und 15min
      \item[meminfo] Speicherauslastung $\rightarrow$ \co{free}
      \item[mounts]  aktuell eingehängte Dateisysteme
      \item[version] Version und Übersetzungsdatum des Kernels
    \end{description}
  \end{block}
\end{frame}

\subsection{Paketverwaltung der verfügaren Programme}
\begin{frame}{Paketverwaltung}
  \begin{block}{Was sind Pakete?}
    \begin{itemize}
      \item Enthalten Dateien und Befehle zur Installation 
      \item Aufteilungen in ``Laufzeitpakete'', Entwicklungspakete, Dokumentation (u. feiner möglich)
      \item unterschliedliche Versionen
      \item Systemweit (de)installierbar 
    \end{itemize}
  \end{block}
  \begin{block}{Paketverwaltungssysteme}
    \begin{itemize}
      \item ``Paketdatenbank'': Informationen zu installierten Paketen\\ $\rightarrow$ \co{/var/lib/\{rpm,dpkg\}}
      \item ``Repositorities'':  enthalten Daten zu verfügbaren Paketen
%       \item Programme: \co{rpm} und \co{dpkg}
    \end{itemize}
  \end{block}
\end{frame}
\begin{frame}{Zwei Geschmacksrichtungen von Paketen}
  \begin{columns}
    \begin{column}{0.52\textwidth}
      \begin{block}{RPM - \textbf{R}ed Hat \textbf{P}ackage \textbf{M}anager}
        \begin{itemize}
          \item von Red Hat entwickelt
          \item \co{rpm}
          \item auflisten mit \co{rpm~--query~--all}
          \item Komfortabler: \co{yum} bzw. \co{zypper} 
        \end{itemize}
      \end{block}
    \end{column}
    \begin{column}{0.48\textwidth}
      \begin{block}{deb}
        \begin{itemize}
          \item von Debian GNU/Linux
          \item \co{dpkg}
          \item auflisten mit \co{dpkg~--list}
          \item Komfortabler: Synaptic, Aptitude
        \end{itemize}
      \end{block}
    \end{column}
  \end{columns}
\end{frame}


\section{Benutzerverwaltung -- Teil I}
\subsection{Grundlagen}
\begin{frame}{Grundlagen}
  \begin{block}{Warum verschiedene Benutzer?}
    \begin{itemize}
      \item gerechtes Aufteilen von Ressourcen (Speicherplatz, \dots)
      \item ``Privatssphäre'': Nicht jeder sollte alles lesen können
      \item keine unauthorisierten Eingriffe ins System
    \end{itemize}
  \end{block}
  \begin{block}{Benutzerkonten}
    \begin{itemize}
      \item Wichtigstes Konto: root \\$\rightarrow$ Administratorkonto (UID=0)
      \item normale Nutzer haben nur vollen Zugriff auf eigene Dateien
      \item Eindeutige Nutzernummer (\textbf{UID}) mit dazugehörigen \textbf{Benutzernamen}
    \end{itemize}
  \end{block}
\end{frame}

\begin{frame}{Gruppen}
  \begin{block}{Was sind Gruppen}
    \begin{itemize}
      \item eindeutige Unterscheidung durch Gruppen-ID (GID)
      \item Jeder Nutzer hat eine primäre Gruppe
            und kann in weiteren sekundären (zusätzlichen) Gruppen Mitglied sein
      \item Nichtmitglieder ggf. durch Gruppenpasswort authentifiziert
    \end{itemize}
  \end{block}
\end{frame}

\begin{frame}{Pseudobenutzer und Natürliche Personen}
  \begin{block}{Über UIDs}
    \begin{itemize}
      \item \co{root}-Rechte verknüpft an die UID=0
      \item Nutzer für spezielle Systemaufgaben (z.B. apache, mail, \dots)
      \item Normale Nutzer bekommen UIDs ab $1000$
      \item Nutzer nutzt Nutzername
      \item System nutzt UID
    \end{itemize}
  \end{block}
\end{frame}

\begin{frame}{Nützliche Programme für Nutzerinformationen}
  
  \begin{block}{\co{id} -- Gruppen und IDs}
    \begin{itemize}
      \item \co{\$~id~Nutzer}
      \item Ausgabe von Nutzer, primäre Gruppe und sek. Gruppen (ID und Name)
    \end{itemize}
  \end{block}
    \begin{block}{\co{last} -- Die letzte Anmeldung}
    \begin{itemize}
      \item \co{\$~last~Nutzer}
      \item gibt die Anmeldedaten von Nutzern aus
      \item Daten aus \co{/var/log/wtmp}
    \end{itemize}
  \end{block}
\end{frame}


\subsection{Dateien der Benutzer- und Gruppenverwaltung}

\begin{frame}{\co{/etc/passwd}}
  \begin{center}
    \colorbox{yellow}{\co{Name:Kennwort:UID:GID:GECOS:Heimatverzeichnis:Shell}}
  \end{center}
  \begin{block}{}
    \begin{description}
      \item[Name]        Textueller Nutzername, Login
      \item[Kennwort]    Passwort-Hash 
      \item[UID]         Numerische Kennung des  Nutzers
      \item[GID]         Numerische Gruppenkennung der primären Gruppe
      \item[GECOS]       Kommentarfeld
      \item[Heimatverzeichnis] Nutzerverzeichnis
      \item[Shell]       Login-Shell
    \end{description}
  \end{block}
\end{frame}

\begin{frame}{Details von \co{/etc/passwd} I}
  \begin{block}{Name}
    \begin{itemize}
      \item Umlaute, Satzzeichen und reine Zahlen problematisch
    \end{itemize}
  \end{block}
  \begin{block}{Kennwort}
    \begin{itemize}
      \item gehashtes Kennwort
      \item ``x'' Passwort in Schattendatei
      \item ``!'' Anmeldung gesperrt
    \end{itemize}
  \end{block}
\end{frame}

\begin{frame}{Details von \co{/etc/passwd} II}
  \begin{block}{GECOS}
    \begin{itemize}
      \item \emph{General Electric Comprehensive Operating System}
      \item Informationen wie vollständiger Name, Zimmer- oder Telefonnummer
      \item eher informativen Charakter
    \end{itemize}
  \end{block}
  \begin{block}{Heimatverzeichnis}
    \begin{itemize}
      \item Verzeichnis für Nutzerdateien
      \item normalerweise in \co{/home/Nutzername}
      \item aktuelles Verzeichnis nach der Anmeldung
    \end{itemize}
  \end{block}
\end{frame}

\begin{frame}{Schattenkennwörter}
  \begin{alertblock}{Problem}
    \co{/etc/passwd} muss für alle lesbar sein, 
    damit die Zuordnung von UID$\leftrightarrow$Nutzername funktioniert
  \end{alertblock}

  \begin{exampleblock}{Lösung}
%     \begin{itemize}
%       \item 
      Verschlüsselte Kennwörter als ``Schattenkennwörter'' in einer Datei
        mit Lesezugriff nur für privilegierte Prozesse und \co{root}\\
        $\rightarrow$ \co{/etc/shadow}
%     \end{itemize}
  \end{exampleblock}
\end{frame}

\begin{frame}{\co{/etc/shadow}}
  \vspace{-2em}
  \begin{center}
  \hspace{-1.4em}\colorbox{yellow}{\co{Name:Kennwort:Änderung:Min:Max:Warn:Frist:Sperre:Reserve}}
  \end{center}
  \begin{block}{}
    \begin{description}
      \item[Name]        Textueller Nutzername, Login
      \item[Kennwort]    verschlüsseltes Passwort
      \item[Änderung]    Datum des letzten Passwortwechsels
      \item[Min]	 Mindestalter für ein Passwort
      \item[Max]	 Höchstalter für ein Passwort
      \item[Warn]	 Tage vor Max für eine Warnung
      \item[Frist]	Tage nach Max, für eine mögliche Anmeldung
      \item[Sperre]	Datum, ab dem der Zugang gesperrt wird
      \item[Reserve]	\dots
    \end{description}
  \end{block}
\end{frame}

\begin{frame}
  \begin{center}
    \Huge Fortsetzung folgt!
  \end{center}
\end{frame}


% \begin{frame}{\co{/etc/group}}
%   \begin{center}
%     \colorbox{yellow}{\co{Gruppenname:Kennwort:GID:Mitglieder}}
%   \end{center}
%   \begin{block}{}
%     \begin{description}
%       \item[Gruppenname] Textueller Name der Gruppe
%       \item[Kennwort]    Wird beim Wechseln der Gruppe mit \co{newgrp} abgefragt, \co{*} und \co{x} wie bei \co{passwd} 
%       \item[GID]         Numerische Gruppenkennung
%       \item[Mitglieder]  Durch Komma getrennte Liste der Mitglieder (Sekundäre Gruppenmitglieder)
%     \end{description}
%   \end{block}
%     Kennwort und Mitglieder sind optional
% \end{frame}
% \begin{frame}{\co{/etc/gshadow}}
% %   \co{}
%   \begin{block}{Gruppenpasswörter}
%     \begin{itemize}
%       \item werden im Allgemeinen nicht verwendet
%       \item \co{/etc/gshadow}
%     \end{itemize}
%   \end{block}
% \end{frame}

% % TODO anpassen
% \begin{frame}{Prüfungsziele}
%   \begin{alertblock}{3.3 Befehle in ein Skript umwandeln}
%     \begin{description}
%       \item[Gewicht]  4
%       \item[Beschreibung] Wiederkehrende Befehle in einfachen Skripte umwandeln
%     \end{description}
%        Wissensgebiete\\ 
%         \begin{itemize}
%           \item Grundlagen der Textverarbeitung und von Shell-Skripten
% %           \item einfaches Globbing und Quoting
%           \item \co{/bin/sh}
%           \item Variablen und Argumente
%           \item Rückgabewerte
%           \item \co{for}-Schleifen (und \co{while}-Schleifen)
%           \item \co{echo}
% %           \item 
%         \end{itemize}
%   \end{alertblock}
% \end{frame}
%%%%%%%%%%%%%%%%%%%%%%%%%% next time...
\begin{frame}[plain]
  \begin{alertblock}{Nächste Vorlesung}
    \textbf{Termin:} Morgen, 25.10.2015 9:20 Uhr\\
    \textbf{Thema:} Kapitel 9 - Benutzerverwaltung \\
    \textbf{Lehrbuchkapitel:} 
    \begin{itemize}
      \item LXES 13 Benutzerverwaltung
      \item LXES 14 Zugriffsrechte
    \end{itemize}
  \end{alertblock}
\end{frame}



\materialframe
\versionframe

\end{document}
