\documentclass[aspectratio=43]{beamer}
\usepackage{../../resources/agdsncourse} 

\title[Linux Essentials  - Kapitel 7 - Archivieren]{Linux Essentials}
\subtitle{Kapitel 7 - Archivieren}
\author{Hagen Eckert}
\date{24. Oktober 2015}


\begin{document}

% Start Folie
\logoframe

%Title
\frame{\titlepage}

%Gliederung
\setcounter{tocdepth}{1}
\section[Gliederung]{}
\frame{\tableofcontents}

% Inhalt
%%%%%%%%%%%%%%%
%%%%%%%%%%%%%%%%%%%%%%%%%%%%
\section{Archivieren}
\subsection{Übersicht}
\begin{frame} 
	\begin{block}{Übersicht} 
	\begin{itemize}
	\item Zusammenfassung von Datenstrukturen
	\item Damit geeignet für lineare Speichermedien wie Bänder 
	\item Neben dem eigentlichen Inhalt werden auch Metadaten abgelegt  (Pfad, Datum, ...)
	\item Zusätzlich kann eine Komprimierung erfolgen
	\end{itemize}
	\end{block}


\end{frame}


\subsection{tar}
\begin{frame} 
	\begin{block}{\co{tar}} 
	\begin{itemize}
	\item Bedeutet \textbf{t}ape \textbf{ar}chive
	\item Ist auf fast jedem Linux/Unix zu finden
	\item Kann auch das gesamte Archive verdichten
	\end{itemize}
	\end{block}
\end{frame}

\begin{frame} 
	\begin{exampleblock}{Usage}
	\co{tar <Option> <Datei>|<Verzeichnis>}
	\end{exampleblock}

	\begin{exampleblock}{Options I}
	\begin{itemize}
	\item[-c] erzeugt (\textbf{c}reate) ein neues Archiv
	\item[-f] Datei erzeugt oder liest das Archiv von einer Datei|Gerät
	\item[-M] bearbeitet ein tar -Archiv, das sich über mehrere Datenträger erstreckt (\textbf{m}ulti-volume archive)
	\item[-r] hängt Dateien an das Archiv an
	\item[-t]  zeigt den Inhalt (\textbf{t}able of contents) des Archivs
	\item[-u] hängt die neue Version von geänderten, schon archivierten Dateien an das Archiv an
	\end{itemize}
	\end{exampleblock}
\end{frame}

\begin{frame} 
	\begin{exampleblock}{Options II}
	\begin{itemize}

	\item[-v] ausführlicher Modus (\textbf{v}erbose, geschwätzig); zeigt auf dem Bildschirm an,
was gerade geschieht
	\item[-x]
 Auslesen (\textbf{e}xtract) der gesicherten Dateien
	\item[-z] komprimiert oder dekomprimiert das Archiv mit gzip
\item[-Z] komprimiert oder dekomprimiert das Archiv mit compress (unter Linux normalerweise nicht vorhanden)
\item[-j] komprimiert oder dekomprimiert das Archiv mit bzip2
	\end{itemize}
	\end{exampleblock}

	
\end{frame}


\subsection{gzip}
\begin{frame} 
	\begin{block}{\co{gzip}} 
	\begin{itemize}
	\item Kurz für \textbf{G}NU \textbf{zip}
	\item Bündelt datein nicht sondern komprimiert nur
	\item Nicht direkt zip der Windowswelt zu vergleichen
	\item Nutzt den DEFLATE Algorithmus
	\end{itemize}
	\end{block}
\end{frame}

\begin{frame} 
	\begin{exampleblock}{Usage}
	\co{gzip <Option> <Datei>
	}
	\end{exampleblock}

	\begin{exampleblock}{Options I}
	\begin{itemize}
	\item[-c] schreibt die komprimierte Datei auf die Standardausgabe, anstatt die Datei zu ersetzen
	\item[-d] dekomprimiert die Datei (gunzip arbeitet wie gzip -d )
	\item[-l] zeigt (\textbf{l}ist) wichtige Verwaltungsinformationen der komprimierten Datei, wie Dateiname, ursprüngliche und gepackte Größe an

	\end{itemize}
	\end{exampleblock}
\end{frame}

\begin{frame} 
	\begin{exampleblock}{Options II}
	\begin{itemize}
	\item[-r] packt auch Dateien in darunterliegenden Verzeichnissen (\textbf{r}ecursive)
	\item[-S] verwendet anstelle von .gz die angegebene Endung (\textbf{S}uffix)
	\item[-v] gibt den Namen und den Kompressionsfaktor für jede Datei aus 
	\item[-n] $n= 1...9$ gibt einen Kompressionsfaktor an
	\newline \co{-1} (oder \co{--fast} ) arbeitet am schnellsten, komprimiert aber nicht so gründlich
	\newline \co{-9} (oder \co{--best}) die beste Komprimierung um den Preis höherer Laufzeit liefert
	\newline voreingestellt ist \co{-6}
	\end{itemize}
	\end{exampleblock}

	
\end{frame}


\subsection{bzip2}
\begin{frame} 
	\begin{block}{\co{bzip2}} 
	\begin{itemize}
	\item Noch ein wenig Platzsparender
	\item Verwendet eine Burrows-Wheeler-Transformation
	\item Gleiche Optionen wie \co{gzip}
	\item Aber: Zahlen geben die Blockgröße und nicht die Qualität an 
	\newline Default: \co{-9} entspricht 900 KB 
	\end{itemize}
	\end{block}
\end{frame}

\subsection{zip}
\begin{frame} 
	\begin{block}{\co{zip}} 
	\begin{itemize}
	\item Das bekannte ZIP aus der Windowswelt
	\item Sehr vielfältig
	\item Hinweis: \co{unzip} ist ein eigenständiges Programm
	\end{itemize}
	\end{block}
\end{frame}

\begin{frame} 
	\begin{exampleblock}{Usage}
	\co{zip <Optionen> <Archiv> <Dateien>\&|<Verzeichnis>}
	\end{exampleblock}

	\begin{exampleblock}{Options I}
	\begin{itemize}
	\item[-r] packt auch Dateien in darunterliegenden Verzeichnissen (\textbf{r}ecursive)
	\item[-@] ließt die Dateinamen von der Kommandozeile (StdIn)
	\item[-0] nicht komprimiert
	\item[-h2] ausführliche Hilfe
	\end{itemize}
	\end{exampleblock}
\end{frame}

\begin{frame} 
	\begin{exampleblock}{Usage}
	\co{unzip <Optionen>  <Archiv> <Dateien>\&|<Verzeichnis>}
	\end{exampleblock}

	\begin{exampleblock}{Options I}
	\begin{itemize}
	\item[-v] Details zum Archive anzeigen
	\item[-d] Verzeichnis zum entpacken angeben (\textbf{d}irectory)
	\item[-x] folgende Inhalte nicht entpacken
	\item[-o] überschreiben von vorhandenen Dateien (\textbf{o}verride)
	\end{itemize}
	\end{exampleblock}
\end{frame}



\begin{frame}[plain]
\begin{alertblock}{Nächste Vorlesung}
\textbf{Termin:} In 15 Minuten\\
\textbf{Thema:} Kapitel 6 - Mehr Shell \\
\textbf{Lehrbuchkapitel:} 
\begin{itemize}
\item LXES 9 Mehr über die Shell
\end{itemize}
\end{alertblock}
\end{frame}

\materialframe
\versionframe


\end{document}
