\documentclass[aspectratio=43]{beamer}
\usepackage{../../resources/agdsncourse} 



\title[Linux Essentials  - Kapitel 6 - Mehr Shell]{Linux Essentials}
\subtitle{Kapitel 6 - Shell}
\author{Alexander Köhler}
\date{24. Oktober 2015}
\begin{document}

% Start Folie
\logoframe

%Title
\frame{\titlepage}

%Gliederung
\setcounter{tocdepth}{1}
\section[Gliederung]{}
\frame{\tableofcontents}


%%%%%%%%%%%%%%% der Inhalt
\section{Mehr Spaß mit der Shell}
\subsection{Internes und Externes der Shell}
\begin{frame}{Mehr über Kommandos herausfinden}
%   \begin{columns}
%     \begin{column}{0.5\textwidth}
      \begin{block}{Allgemeiner Ablauf zum Ausführen eines Befehls}
%         \begin{enumerate}
%           \item
          Eingabe des Kommandos:
          \begin{enumerate}
            \item interner Befehl $\rightarrow$ Shell\\
              feststellbar mit \co{type}
            \item externer Befehl 
              $\rightarrow$ Alle Verzeichnisse in \co{\$PATH} werden nacheinander
              durchsucht
          \end{enumerate}
%         \end{enumerate}
      \end{block}
%     \end{column}
%     \begin{column}{0.5\textwidth}
  \begin{columns}
    \begin{column}{0.3\textwidth}
      \begin{block}{Typ feststellen}
            \co{type kommando} \\$\rightarrow$  Befehlstyp (intern/extern)
      \end{block}
    \end{column}
    \begin{column}{0.75\textwidth}
      \begin{block}{Externe Befehle}
            \begin{itemize}
              \item \co{which} $\rightarrow$ gibt exakten Kommandopfad aus
              \item \co{whereis} liefert weitere interessante Dateipfade (Konfiguration, Manual-Seiten)
            \end{itemize}
      \end{block}
    \end{column}
  \end{columns}
\end{frame}

\begin{frame}{Variablen}
  \begin{columns}
    \begin{column}{0.435\textwidth}
      \begin{block}{Shell-Variablen}
        \begin{itemize}
%           \item Geldsack im eigenen Zimmer
          \item stehen nur der aktuellen Shell zur Verfügung
          \item explizite Weitergabe an Kindsprozess nötig
          \item erstellen: \co{variable=wert}
          \item alle anzeigen: \co{set}
          \item komplett löschen: \co{unset~variable}
        \end{itemize}
      \end{block}
    \end{column}
    \begin{column}{0.65\textwidth}
       \begin{block}{Umgebungsvariablen}
            \begin{itemize}
%             \item Geschirr in der WG-Küche
              \item stehen der aktuellen Shell sowie ihren Kindsprozessen zur Verfügung
              \item sind Shell-Variablen der aktuellen Shell
              \item von Shell- zur Umgebungsvariablen: \co{export~variable=wert}
              \item anzeigen aller Variablen: \co{export} (oder \co{env})
              \item entfernen: \co{export~-n~variable} (verbleibt als reine Shell-Variable)
            \end{itemize}
        \end{block}
    \end{column}
  \end{columns}
  $\rightarrow$ Inhalt einer Variable wird mit vorangestellten ``\$'' referenziert
\end{frame}
    
\begin{frame}{Einige wichtige Variablennamen}
  \begin{tabular}{cp{0.8\textwidth}}
    Variable         & Bedeutung\\
    \hline
    \co{PWD}         & Aktuelles Arbeitsverzeichnis\\
    \co{PS1}         & Angaben zum Prompt (Text vor dem \$ der Shell)\\
    \co{UID}         & Benutzerkennung (ID)\\
    \co{HOME}        & Pfad des Heimatsverzeichnis\\
    \co{PATH}        & \textbf{Suchpfade für Befehle}, Trenner der Pfade ist ``:''\\
    \co{LOGNAME}     & Benutzername\\
    \co{TZ}          & manuell gesetzte Zeitzone \\
  \end{tabular}
\end{frame}


\subsection{Kontrollstrukturen für die Shell-Programmierung}
\begin{frame}{Rückgabewerte}
  \begin{columns}
    \begin{column}{0.4\textwidth}
      \begin{block}{Arten der Rückgabewerte}
        \begin{itemize}
          \item Jedes Kommando liefert nach dem Aufruf einen Rückgabewert zurück
          \item Letzter Rückgabewert steht in \co{\$?}
          \item Erfolg: ``0''
          \item Misserfolg: alles andere\\
               (Es gibt nur eine Art des Erfolgs!)
        \end{itemize}
      \end{block}
    \end{column}
    \begin{column}{0.7\textwidth}
       \begin{block}{Einfache Verknüpfungen von Kommandos}
            \begin{itemize}
              \item Kommandotrenner: \co{;}
              \item Und-Verknüpfung: \co{\&\&}\\
                    $\rightarrow$ zweites Kommando wird nur ausgeführt, wenn das erste Kommando
                    erfolgreich war
              \item Oder-Verknüpfung: \co{||}\\
                    $\rightarrow$ zweites Kommando wird nur ausgeführt, wenn das erste Kommando
                    nicht erfolgreich war
              \item Rückgabewert der Verknüpfung ist die log. Verknüpfung der beiden Kommandos
            \end{itemize}
        \end{block}
    \end{column}
  \end{columns}
\end{frame}

\begin{frame}{Fallunterscheidungen}
  \begin{columns}
    \begin{column}{0.5\textwidth}
      \begin{block}{Testen mit \co{test}}
        \begin{itemize}
          \item Überprüft Argument(e) auf versch. Eigenschaften
          \item z.B. Vergleiche der Zahlenwerte bzw. Zeichenketten, Dateieigenschaften
          \item Kurzform: \co{[ Arg1 \dots~]}
        \end{itemize}
      \end{block}
    \end{column}
    \begin{column}{0.55\textwidth}
       \begin{block}{Komplexere Fallunterscheidungen}
            \begin{itemize}
              \item bequeme und leserliche Abfrage von verschiedenen Fällen
%               \item wie bei anderen Programmiersprachen auch
            \end{itemize}
            \co{\textbf{if} [ Bedingung1 ]\\
                \textbf{then}\\
                ~~echo erste Wahl\\
                \textbf{elif} [ Bedingung2 ]\\
                ~~echo zweite Wahl\\
                \textbf{else}\\
                ~~echo nichts passt\\
                \textbf{fi}}
        \end{block}
    \end{column}
  \end{columns}
\end{frame}

\begin{frame}{Zweimal Schleifen}
  \begin{columns}
    \begin{column}{0.5\textwidth}
      \begin{block}{\co{for}-Schleife}
        \begin{itemize}
          \item Abarbeiten einer Liste
        \end{itemize}
        \co{\$ for i in \{a..e\} \\
            > do \\
            > ~~echo i = \$i\\
            > done}
      \end{block}
    \end{column}
    \begin{column}{0.6\textwidth}
       \begin{block}{\co{while}-Schleife}
            \begin{itemize}
              \item läuft solange eine Bedingung erfüllt ist
            \end{itemize}
        \co{\$ while testbedingung \\
            >~do \\
            >~~~echo~Bedingung~noch~erfüllt.\\
            >~done}
        \end{block}
    \end{column}
  \end{columns}
\end{frame}

% 
% \begin{frame}{}
%   \begin{columns}
%     \begin{column}{0.5\textwidth}
%       \begin{block}{}
%         \begin{itemize}
%           \item 
%         \end{itemize}
%       \end{block}
%     \end{column}
%     \begin{column}{0.5\textwidth}
%        \begin{block}{}
%             \begin{itemize}
%               \item 
%             \end{itemize}
%         \end{block}
%     \end{column}
%   \end{columns}
% \end{frame}

\subsection{Nützliche Befehle für die Shell}
\begin{frame}{}
  \begin{columns}
    \begin{column}{0.5\textwidth}
      \begin{block}{Die Shell schlafen schicken}
        \begin{itemize}
          \item \co{\$~sleep~3} $\rightarrow$ Shell schläft für 3 Sekunden
          \item während \co{sleep} aktiv ist, wird kein anderer Befehl ausgeführt
        \end{itemize}
      \end{block}
    \end{column}
    \begin{column}{0.5\textwidth}
       \begin{block}{Wenn die Shell antwortet -- echo}
            \begin{itemize}
              \item \co{\$ echo Irgend ein Text}
              \item nützlich zur Ausgabe von Variableninhalte, z.B. \co{echo~\$PATH}
              \item \co{-n} unterdrückt Zeilenumbruch
            \end{itemize}
        \end{block}
    \end{column}
  \end{columns}
\end{frame}
\begin{frame}{Datum in der Shell}
\begin{columns}
        \begin{column}{0.5\textwidth}
        \begin{block}{\co{date}} 
        \begin{itemize}
        \item wichtiges Tool wenn es um Zeit geht
        \item kann die Softwareuhr setzen (root)
        \item unterschiedlichste Layouts
        \item Zeitzonen\\ $\rightarrow$ \co{/usr/share/zoneinfo}
        \end{itemize}
        \end{block}
        \end{column} 
        
        \begin{column}{0.5\textwidth} 
        \begin{exampleblock}{Beispiele} 
        \begin{itemize}
        \item \co{date}
        \item \co{export TZ=Europe/London}
        \item \co{export TZ=America/Vancouver}
        \item \co{date +\tsq\%d.\,\%B \%Y\tsq}
        \item \co{date +\tsq{}Tag Nr.\,\%j\tsq}
        \item \co{date +\tsq{}KW Nr.\,\%V\tsq}
        \end{itemize}
        \end{exampleblock}
        \end{column} 
\end{columns}
\end{frame}

\begin{frame}{Kommandos aus der Datei}
  \begin{block}{Befehle in einer Datei ausführen}
    \begin{itemize}
      \item \co{\$~bash~shellscript}
      \item Befehle werden in Sub-Shell (Kindsprozess) ausgeführt
      \item Variablenzuweisungen werden nicht an Elternshell weitergegeben!
      \item Alternative: \co{\$~./shellscript}\\
          $\rightarrow$ Shebang (\co{\#!~/bin/bash}) und ausführbar (\co{chmod})
    \end{itemize}
  \end{block}
  \begin{block}{Inhalt von Dateien einlesen}
    \begin{itemize}
      \item \co{\$~source datei} bzw. \co{\$~.~datei}
      \item Befehle und Variablenzuweisungen werden in aktueller Shell ausgeführt!
    \end{itemize}
  \end{block}
\end{frame}

\begin{frame}{Weitere Hilfen in der Shell}
  \begin{itemize}
    \item \taste{Tab}: Autovervollständigung
    \item \taste{Tab}\taste{Tab}: Auflisten aller Eingabemöglichkeiten
    \item \taste{$\uparrow$} und \taste{$\downarrow$}: führere Befehle durchblättern
    \item \taste{Strg}+ \taste{r}: frühere Befehle durchsuchen
    \item \co{\$~history}: listet zuvor eingegebene Befehle auf
    \item \taste{Strg}+ \taste{c}: bricht Eingabe/laufenden Befehl ab
    \item \taste{Strg}+ \taste{d}: Eingabe beenden/Abmelden aus der Shell
%     \item 
  \end{itemize}

\end{frame}



\section{Das Dateisystem}
\subsection{Verschiedene Dateitypen}
\begin{frame}{Dateitypen I}
  \begin{columns}
    \begin{column}{0.5\textwidth}
      \begin{block}{Normale Dateien}
        \begin{itemize}
          \item Textdateien, \dots
          \item div. Programme zum Anlegen
        \end{itemize}
      \end{block}
      \begin{block}{Symbolische Links}
        \begin{itemize}
	  \item  Verweise auf andere Dateien (vgl. Kapitel 4)
	  \item ls-Kennung: \co{l} oder \co{@}
          \item  \co{ln~-s}
        \end{itemize}
      \end{block}
    \end{column}
    \begin{column}{0.5\textwidth}
       \begin{block}{Verzeichnisse}
            \begin{itemize}
	      \item  Tabelle mit Einträgen für Dateinamen $\leftrightarrow$ Inode-Nr.
	      \item ls-Kennung: \co{d} oder \co{/}
              \item  \co{mkdir}
            \end{itemize}
        \end{block}
       \begin{block}{Gerätedateien}
            \begin{itemize}
	      \item entspricht einem Gerät bzw. einem Treiber
	      \item ls-Kennung: \co{c} bzw. \co{b}
              \item \co{mknod}
            \end{itemize}
        \end{block}
    \end{column}
  \end{columns}
\end{frame}
% \begin{frame}{Dateitypen II}
%   \begin{columns}
%     \begin{column}{0.5\textwidth}
%     \end{column}
%     \begin{column}{0.5\textwidth}
%     \end{column}
%   \end{columns}
% \end{frame}
\begin{frame}{Dateitypen II}
  \begin{columns}
    \begin{column}{0.5\textwidth}
      \begin{block}{FIFOs -- Named pipes}
        \begin{itemize}
	  \item unidirektionale Verbindung von Prozessen
          \item vgl. Pipes (\textbar)
          \item ls-Kennung: \co{p} oder \co{\textbar}
          \item \co{mkfifo}
        \end{itemize}
      \end{block}
    \end{column}
    \begin{column}{0.5\textwidth}
       \begin{block}{Unix-Domain-Sockets}
            \begin{itemize}
              \item bidirektionelle Kommunikation von Prozessen
	      \item ls-Kennung: \co{s} oder \co{=}
              \item kein Programm zum Erstellen
            \end{itemize}
        \end{block}
    \end{column}
  \end{columns}
  \begin{exampleblock}{Dateien genauer untersuchen}
    \co{\$~file~dateiname}\\
    $\rightarrow$ Informationen zum Dateityp und Inhalt (normale Dateien)
  \end{exampleblock}

\end{frame}

\subsection{Der Linux-Verzeichnisbaum}
\begin{frame}{}
  \begin{exampleblock}{Idee des Filesystem Hierarchy Standard (FHS)}
    \begin{itemize}
     \item Verzeichnisse in der ersten Hierarchie-Ebene und unter \co{/usr/} werden beschrieben
     \item Unterteilung:
      \begin{itemize}
       \item lokal $\leftrightarrow$ entfernt (übers Netz bezogen)
       \item statisch $\leftrightarrow$ dynamisch (Änderungen vorgesehen)
      \end{itemize}

    \end{itemize}

  \end{exampleblock}
  \begin{alertblock}{Problem}
    Nicht jede Distribution muss sich daran halten
  \end{alertblock}

\end{frame}

\begin{frame}{Alles für dem Systemstart}
  \begin{block}{\co{/boot}}
%     \begin{itemize}
        Kernel (Betriebssystemkern),
       Dateien für Bootloader (z.B. GRUB)
%     \end{itemize}
  \end{block}
  \begin{block}{\co{/bin}}
    Systemprogramme, die zum Systemstart wichtig sind oder ständig gebraucht werden
    (\co{grep}, \co{ln}, \co{date})
  \end{block}
  \begin{block}{\co{/sbin}}
    wie \co{/bin}, aber Systemkonfigurationswerkzeuge, die nur root ausführen soll
  \end{block}
  \begin{block}{\co{/lib} --  shared libaries}
%     \begin{itemize}
%       \item 
      Bibliotheken der Programme aus \co{/bin} und \co{/sbin}
      Kernelmodule (\co{/lib/modules}; Gerätetreiber, Netzwerkprotokolle, \dots)
%     \end{itemize}
  \end{block}
\end{frame}

\begin{frame}{\co{/dev} -- Gerätedateien}
  \begin{block}{Allgemeines}
    \begin{itemize}
     \item Jedes Gerät mit Treiber im System entspricht einer Datei
     \item blockorientiert (Block-Devices): Lesen und Schreiben blockweise
     \item zeichenorientiert (Character-Devices): Lesen und Schreiben einzelne Bytes
    \end{itemize}
  \end{block}
  \begin{block}{Pseudogeräte}
    \begin{description}
      \item[\co{/dev/null}] ``Mülleimer'' für Bytes
      \item[\co{/dev/(u)random}] liefert zufällig generierte Bytes
      \item[\co{/dev/zero}] gibt beliebig viele Nullbytes aus
    \end{description}
  \end{block}
\end{frame}

\begin{frame}{\co{/etc} -- Konfigurationsdateien}
%   \begin{block}{}
%     
%   \end{block}
  \begin{block}{Wichtige Dateien}
    \begin{description}
      \item[\co{/etc/fstab}] Zuordnung bekannter Datenträger mit Einhängepunkten
      \item[\co{/etc/mtab}] Aktuell eingebundene Datenträger
      \item[\co{/etc/inittab}] Systemstart
      \item[\co{/etc/init.d/}] Skripte für den Systemstart
      \item[\co{/etc/hosts}] Zuordnung von Rechner und Namen im kleinen Netzen
      \item[\co{/etc/passwd}] Benutzerdaten (Namen, Gruppen, \dots)
      \item[\co{/etc/issue}] Text, der vor einer Anmeldung angezeigt wird
      \item[\co{/etc/motd}] Nachrichtentext nach der Ameldung
    \end{description}
  \end{block}
\end{frame}

\begin{frame}{\co{/usr} -- Die Unveränderlichen (Dateien)}
  \begin{block}{Idee}
    Enthält alle Dateien, die nicht während Laufzeit geändert werden müssen
  \end{block}
  \begin{block}{Inhalt}
    \begin{description}
     \item[\co{/usr/share/}] Architektur-unabhängige Daten %(Manualseiten, Dokumentationen)
     \item[\co{/usr/share/\{man,info,doc\}}] Dokumentationen, Manualseiten, Hilfeseiten 
     \item[\co{/usr/\{bin,sbin,lib\}/}] Programme und Bibliotheken, die nicht 
					zum Systemstart benötigt werden
     \item[\co{/usr/local/}] Programme, die vom lokalen Admin manuell installiert wurden
     \item[\co{/usr/src/}] enthält ggf. Quellcode von Programmen
    \end{description}
  \end{block}
\end{frame}

\begin{frame}{\co{/var} und \co{/tmp} -- Veränderliche und temporäre Dateien }
  \begin{block}{Idee}
    Enthält alle Dateien, die vom System geändert werden bzw. nur temporär benötigt werden
  \end{block}
  \begin{block}{Inhalt}
    \begin{description}
     \item[\co{/tmp/}] Inhalt sollte bei einem Systemstart gelöscht werden
     \item[\co{/var/}] Dateien, die häufig geändert werden (Drucker-, E-Mailwarteschlangen,\dots)
     \item[\co{/var/log/}] Protokolldateien des Systems (von Syslogd, Klogd, dmesg)
    \end{description}
  \end{block}
\end{frame}

\begin{frame}{\co{/proc} und \co{/sys} -- Pseudodateisysteme}
  $\rightarrow$ nehmen keinen Speicherplatz ein
  \begin{block}{Prozessdateien in \co{/proc}}
    Informationen über alle Prozesse, teilweise auch noch über die Hardware\\
    $\rightarrow$ Details über genaue Dateien nächste Vorlesung
%     \begin{description}
%      \item[\co{/proc/}]
%     \end{description}
  \end{block}
  \begin{block}{Hardwaredateien in \co{/sys}}
    Konsistente Sicht auf die verfügbare Hardware sowie deren Konfiguration 
    (ab Kernel 2.6.)
  \end{block}
\end{frame}

\begin{frame}{Weitere Verzeichnisse}
  \begin{block}{Heimatverzeichnisse}
    \begin{description}
     \item[\co{/home/}] enthält die Heimatverzeichnisse aller normalen Nutzer
     \item[\co{/root/}] Heimatverzeichnis des Systemadministrators (root)
    \end{description}
  \end{block}
  \begin{block}{}
    \begin{description}
     \item[\co{/lost+found/}] Daten, 
     \item[\co{/opt/}] Installationsort von Programmen Drittanbieter
     \item[\co{/srv/}] Webseiten, FTP-Dateien 
     \item[\co{/media/}] Einhängeorte für Datenträger
     \item[\co{/mnt/}] kurzfristiger Einghängeort für Datenträger
    \end{description}
  \end{block}
\end{frame}

\subsection{Aufteilung auf der Festplatte -- Partitionierung}
\begin{frame}{Letzte Punkte zum Umgang mit Partitionen}
  \begin{block}{Partitionierung}
 \begin{itemize}
  \item grundsätzlich nur eine Partition nötig
  \item Aufteilung auf versch. Partitionen vorteilhaft
    \begin{itemize}
     \item Bsp. home: Neuinstallation einfacher
     \item Bsp. \co{/var} bzw. \co{/tmp}: Protokolldateien haben immer Platz
    \end{itemize}
 \end{itemize}
  \end{block}
  \begin{block}{Einhängen von Datenträgern/Partitionen}
    \begin{itemize}
     \item \co{/etc/fstab} enthält Einhängepunkte der Systemteile
     \item manuelles Einhängen mit:\\
      \co{\$~mount /dev/sda3 /mountpoint}
     \item manuelles Aushängen mit:\\
      \co{\$~umount /dev/sda3} bzw. \co{\$~umount /mountpoint}
    \end{itemize}

  \end{block}

\end{frame}


\begin{frame}{Prüfungsziele I}
  \begin{alertblock}{2.1 Erste Schritte auf der Kommandozeile}
    \begin{description}
      \item[Gewicht]  2
      \item[Beschreibung] Grundkenntnisse zur Benutzung der Linux-Kommandozeile
    \end{description}
       Wissensgebiete\\ 
        \begin{itemize}
          \item Shell-Grundkenntnisse
          \item Kommandos eingeben und strukturieren
          \item Mit Optionen arbeiten
          \item Variablen, Globbing, Suchmuster, Quoting
          \item \co{echo}, \co{history}, \co{which}
%           \item  \co{which}
          \item Umgebungsvariable \co{PATH}
          \item Steuerungsoperatoren \co{\&\&}, \co{\textbar\textbar} und \co{;}
        \end{itemize}
  \end{alertblock}
\end{frame}

% TODO anpassen
\begin{frame}{Prüfungsziele II}
  \begin{alertblock}{3.3 Befehle in ein Skript umwandeln}
    \begin{description}
      \item[Gewicht]  4
      \item[Beschreibung] Wiederkehrende Befehle in einfachen Skripte umwandeln
    \end{description}
       Wissensgebiete\\ 
        \begin{itemize}
          \item Grundlagen der Textverarbeitung und von Shell-Skripten
%           \item einfaches Globbing und Quoting
          \item \co{/bin/sh}
          \item Variablen und Argumente
          \item Rückgabewerte
          \item \co{for}-Schleifen (und \co{while}-Schleifen)
          \item \co{echo}
%           \item 
        \end{itemize}
  \end{alertblock}
\end{frame}

\begin{frame}{Prüfungsziele III}
  \begin{alertblock}{4.3 Datenspeicherung}
    \begin{description}
      \item[Gewicht]  3
      \item[Beschreibung] Wo verschiedene Arten von Informationen in einem Linux-System gespeichert werden
    \end{description}
       Wissensgebiete\\ 
        \begin{itemize}
          \item Kernel, Prozesse (\co{/proc})
          \item \co{/lib}, \co{/usr/lib}, \co{/etc}, \co{/var/log}
          \item  syslog, klog, dmesg
          \item Orte der Systemkonfiguration (\co{/etc/})
        \end{itemize}
  \end{alertblock}
\end{frame}
%%%%%%%%%%%%%%%%%%%%%%%%%% next time...
\begin{frame}[plain]
  \begin{alertblock}{Nächste Vorlesung}
    \textbf{Termin:} In 15 Minuten\\
    \textbf{Thema:} Kapitel 7 - Archivieren\\
    \textbf{Lehrbuchkapitel:} 
    \begin{itemize}
      \item LXES 11 Dateien archivieren und komprimieren
      \item LXES 12 Einstieg in die Systemadministration
    \end{itemize}
  \end{alertblock}
\end{frame}


\materialframe
\versionframe

\end{document}
